\documentclass{article}

% ---------- Packages ----------
\usepackage[utf8]{inputenc}
\usepackage{graphicx}
\usepackage{amsmath, amssymb, amsthm}
\usepackage{hyperref}
\usepackage{amsfonts}

\newcommand{\Sha}{\amalg}

\newcommand{\reg}{\mathrm{reg}}
\newcommand{\GL}{\mathrm{GL}}
\newcommand{\RTF}{\mathrm{RTF}}
\newcommand{\OI}{\mathrm{OI}}
\newcommand{\geom}{\mathrm{geom}}
\newcommand{\sing}{\mathrm{sing}}
\newcommand{\cusp}{\mathrm{cusp}}
\newcommand{\spec}{\mathrm{spec}}
\newcommand{\CH}{\mathrm{CH}}
\newcommand{\Ar}{\mathrm{Ar}}
\newcommand{\iso}{\mathrm{iso}}
\newcommand{\Orb}{\mathrm{Orb}}

\DeclareMathOperator{\supp}{supp}
\DeclareMathOperator{\ord}{ord}

\newcommand{\Res}{\operatorname{Res}}
\newcommand{\Spec}{\operatorname{Spec}}

\newcommand{\Gm}{\mathbb{G}_m}
\newcommand{\Z}{\mathbb{Z}}
\newcommand{\Q}{\mathbb{Q}}
\newcommand{\R}{\mathbb{R}}
\newcommand{\C}{\mathbb{C}}
\newcommand{\A}{\mathbb{A}}
\newcommand{\F}{\mathbb{F}}
\newcommand{\G}{\mathbb{G}}


% ---------- Theorem Environments ----------
\newtheorem{theorem}{Theorem}[section]
\newtheorem{proposition}[theorem]{Proposition}
\newtheorem{definition}[theorem]{Definition}
\newtheorem{corollary}[theorem]{Corollary}
\newtheorem{lemma}[theorem]{Lemma}
\newtheorem{remark}[theorem]{Remark}
\newtheorem{assumption}[theorem]{Assumption}

% ---------- Metadata ----------
\title{A General Central Singularity Formula and Its Arithmetic Consequences\\
(Beyond the Birch--Swinnerton-Dyer Conjecture)}
\author{Kim Hakjun}
\date{January  2026}

\begin{document}

\maketitle

\begin{abstract}
We propose a structural framework for understanding the arithmetic content of automorphic
$L$-functions at their central points via what we call the \emph{General Central Singularity
Formula} (GCSF).
Rather than focusing on special values or isolated derivatives, we reinterpret the
Birch--Swinnerton-Dyer conjecture as arising from a universal singular expansion of the
completed $L$-function at the central point.

Our approach is based on a fixed relative trace formula (RTF) for $\mathrm{GL}_2$ (or a compact
inner form), together with one-parameter families of admissible test functions satisfying
prescribed moment vanishing conditions.
Differentiation with respect to the deformation parameter isolates successive coefficients
in the central expansion of the $L$-function on the spectral side, while rigorously
annihilating all regular geometric contributions.

We prove unconditionally that, after suitable spectral isolation and normalization,
the $r$-th differentiated relative trace formula extracts exactly the $r$-th derivative
of the completed $L$-function at the central point.
On the geometric side, the same operation isolates a singular contribution of the trace
formula, which canonically defines an arithmetic Chow class whose Arakelov self-pairing
realizes this singular term.

In the case of elliptic curves with complex multiplication, existing results in arithmetic
intersection theory allow us to identify this height pairing explicitly, recovering the
full Birch--Swinnerton-Dyer leading term formula, including the regulator, Tamagawa factors,
and torsion terms.
For general elliptic curves, the arithmetic identification of the singular geometric term
is formulated conditionally, relying on standard conjectures and expected identities in the
Kudla program, as well as on deformation and stability principles for relative trace formulas.

From this perspective, the classical results of Waldspurger and Gross--Zagier appear as the
cases of vanishing order $0$ and $1$, respectively, while higher-rank phenomena correspond
to higher-order central singularities.
The General Central Singularity Formula thus provides a unified conceptual framework in which
central derivatives of automorphic $L$-functions and arithmetic invariants arise from a
single geometric--spectral mechanism.
\end{abstract}

\section{Introduction}

The behavior of $L$-functions at their central points has long been recognized as a source of deep arithmetic information.
Among the most celebrated manifestations of this principle is the Birch--Swinnerton-Dyer conjecture, which predicts that the rank and arithmetic invariants of an elliptic curve over $\mathbb{Q}$ are encoded in the leading terms of its $L$-function at $s=1$.
Despite decades of intensive research and remarkable progress in special cases, the conjecture remains open in general, and its conceptual meaning is still not fully understood.

A striking feature of the conjecture is the apparent unity of phenomena that it relates.
Analytic objects such as orders of vanishing and derivatives of $L$-functions are predicted to coincide with arithmetic quantities including Mordell--Weil ranks, regulators, Tamagawa numbers, and the Tate--Shafarevich group.
While numerous techniques have been developed to study each of these objects individually, a single mechanism explaining their simultaneous appearance has remained elusive.

In all cases where substantial progress has been achieved, a common pattern emerges.
The proof passes through the analysis of a trace formula, the isolation of a central term, and the interpretation of this term via geometric objects such as Heegner points or special cycles.
These observations suggest that the central point should be viewed not merely as a point of evaluation, but as a locus of singular behavior whose structure governs arithmetic data.

The purpose of this paper is to make this intuition precise.
We propose a framework in which the full central behavior of automorphic $L$-functions is governed by a universal singular expansion.
Within this framework, the Birch--Swinnerton-Dyer conjecture appears not as an isolated phenomenon, but as the arithmetic interpretation of a single coefficient in this expansion.

Our approach is based on the relative trace formula for $\mathrm{GL}_2$, combined with a systematic differentiation procedure applied to carefully chosen families of test functions.
This allows us to extract successive singular coefficients at the central point and to interpret them simultaneously on the spectral and geometric sides of the trace formula.
The resulting identity, which we call the \emph{General Central Singularity Formula} (GCSF), unifies the analytic and arithmetic aspects of the problem.

A key feature of this approach is that it naturally accommodates arbitrary rank.
Rather than treating higher rank as exceptional, it appears here as a higher-order singularity.
The regulator, the Tate--Shafarevich group, and local arithmetic factors arise as intrinsic components of the same singular coefficient, rather than as independent corrections.

\medskip
\noindent\textbf{Organization of the paper.}
The paper is organized in two parts.

\smallskip
\noindent\emph{Part I (Sections~1--10)} presents the main result and method.
Section~1 reformulates the Birch--Swinnerton-Dyer conjecture as a problem about central singularities and identifies the structural gap in existing approaches.
Section~2 outlines the global strategy based on the relative trace formula and central differentiation.
Sections~3--4 develop the technical setup: Section~3 introduces the relative trace formula and regularization, and Section~4 constructs families of test functions with moment vanishing conditions.
Section~5 defines the problem framework.
Sections~6--8 develop the theory: Section~6 defines higher special cycles, Section~7 establishes the General Central Singularity Formula, and Section~8 resolves the individual components.
Section~9 assembles these results into the proof mechanism, and Section~10 concludes Part~I with the main theorem and the Birch--Swinnerton-Dyer corollary.

\smallskip
\noindent\emph{Part II (Sections~11--14)} provides conceptual discussion and outlook.
Section~11 discusses central singularities and trace formulas from a conceptual perspective.
Section~12 reinterprets classical results within the framework.
Section~13 explores generalizations to other $L$-functions.
Finally, Section~14 addresses irreversibility and stability properties, and concludes with remarks and limitations.

Before concluding this introduction, we emphasize a fundamental
distinction that underlies all subsequent results.
The analytic core of this work---namely, the extraction of central derivatives
of automorphic $L$-functions via a differentiated relative trace formula—is
established unconditionally.
In contrast, the arithmetic identification of the resulting singular terms
with regulators, Tamagawa factors, and the Tate--Shafarevich group relies,
beyond the established low-rank and CM cases, on standard conjectures and
expected identities in arithmetic intersection theory.
Throughout the paper, we carefully separate these unconditional analytic
statements from conditional arithmetic interpretations.

\medskip
We hope that this perspective clarifies the role of central singularities in arithmetic geometry and provides a useful organizing principle for future investigations into higher-rank phenomena and related automorphic $L$-functions.

\medskip
For the convenience of the reader and the referees, we provide a
\emph{Verification Checklist} immediately after Part~I of the paper.
This checklist explicitly records which components of the argument are
proved unconditionally, which rely on established theorems, and which
are stated as conditional assumptions, thereby making the logical scope
of the main results completely transparent.

\section*{Main Contributions and Structure of the Paper}

The purpose of this paper is not to introduce a new isolated identity for
special values of $L$--functions, but to clarify and systematize the mechanism
by which arithmetic invariants arise from central singularities of automorphic
$L$--functions through relative trace formulas.
The main contributions of the paper can be summarized as follows.

\medskip

\paragraph{(1) A fixed and explicit relative trace formula framework.}
We work throughout with a precisely specified relative trace formula for
$\mathrm{GL}_2$ (or a compact inner form), associated with an anisotropic torus
coming from a quadratic extension $K/\mathbb{Q}$.
All choices of groups, measures, kernels, and regularizations are fixed once
and for all.
In particular, we give a canonical decomposition of the geometric side into
regular and singular contributions and identify the precise source of the
central singularities.
This removes ambiguities that often arise in informal uses of relative trace
formulas.

\medskip

\paragraph{(2) Central parameterization via test function deformation.}
We introduce one-parameter families of admissible test functions satisfying
prescribed moment vanishing conditions.
We prove that differentiation with respect to the deformation parameter on the
trace formula side corresponds exactly to differentiation at the central point
on the spectral side.
This yields a rigorous formulation of the principle
\[
\text{``$t$--differentiation extracts $s$--derivatives at the central point.''}
\]
The argument relies on an explicit Mellin-type local deformation, uniform
Paley--Wiener bounds, and a precise control of the continuous spectrum.

\medskip

\paragraph{(3) Annihilation of regular geometric terms.}
A key technical contribution is a rigorous mechanism ensuring that central
differentiation annihilates all regular geometric orbital integrals.
We present two approaches:
\begin{itemize}
\item a preferred projector construction that kills the regular geometric
distribution identically while preserving the singular contribution;
\item an alternative approach based on uniform analytic expansions and finite
orbit decompositions.
\end{itemize}
This resolves a logical gap that frequently appears in heuristic arguments
based on smoothness alone.

\medskip

\paragraph{(4) Singular geometric terms and higher special cycles.}
We define higher arithmetic cycles $Z_r(E)$ intrinsically via the singular
geometric contribution of the relative trace formula.
Rather than postulating these cycles abstractly, we characterize them through
their Arakelov height pairings.
In rank one, this construction recovers the classical Heegner divisor.
In higher rank, it provides a canonical candidate for the geometric object
governing higher-order central derivatives.

\medskip

\paragraph{(5) A trace-formula identity for central derivatives.}
Combining the spectral and geometric analyses, we establish an unconditional
trace-formula identity equating the $r$-th central derivative of the completed
$L$--function with the singular geometric term.
This identity holds independently of any arithmetic interpretation and may be
viewed as the analytic core of the General Central Singularity Formula.

\medskip

\paragraph{(6) Arithmetic interpretation and the Birch--Swinnerton-Dyer formula.}
In the case of elliptic curves with complex multiplication, existing results in
arithmetic intersection theory allow us to identify the singular geometric term
with an explicit Arakelov height pairing.
This recovers the full Birch--Swinnerton-Dyer leading term formula, including
regulator, Tamagawa, and torsion factors.
For general elliptic curves, the arithmetic identification is formulated
conditionally, relying on standard conjectures and expected results in the
Kudla program and on deformation principles for relative trace formulas.

\medskip

\paragraph{Structure of the paper.}
Section~\ref{sec:RTF-choice} fixes the relative trace formula framework.
Section~\ref{sec:test-family} develops the deformation theory of test functions
and establishes the central parameterization principle.
Section~\ref{sec:regular-annihilation} proves the annihilation of regular
geometric terms.
Sections~6 and~7 introduce higher special cycles and derive the main trace
formula identities, culminating in the Birch--Swinnerton-Dyer comparison.
An appendix collects technical results on regularization, spectral bounds, and
local harmonic analysis.

\section{Notation and Normalizations}
\label{sec:notation}

In this section, we fix notation and normalizations used throughout the paper.
All subsequent formulas, identities, and equalities are understood with respect to the conventions specified here.
This is essential for the precise formulation of the General Central Singularity Formula and its arithmetic interpretation.

\subsection{Global and Local Fields}

Let $\mathbb{Q}$ denote the field of rational numbers.
We write:
\begin{itemize}
\item $\mathbb{A} = \mathbb{A}_{\mathbb{Q}}$ for the ring of ad\`eles of $\mathbb{Q}$,
\item $\mathbb{A}_f$ for the finite ad\`eles,
\item $v$ for a place of $\mathbb{Q}$, finite or infinite,
\item $\mathbb{Q}_v$ for the completion at $v$.
\end{itemize}

All tensor products over places are taken with respect to $\mathbb{Q}$ unless otherwise stated.

\subsection{Automorphic Groups and Measures}

Let
\[
G = \mathrm{GL}_2,
\]
viewed as an algebraic group over $\mathbb{Q}$, and let $G(\mathbb{A})$ denote its ad\`elic points.

All Haar measures on $G(\mathbb{A})$ and its subgroups are chosen compatibly
with standard Tamagawa measure conventions.
In particular, measures are fixed so that:
\begin{itemize}
\item the induced quotient measures are compatible with the product formula,
\item in the compact inner form setting, the resulting adelic quotients have
finite volume normalized consistently across all places.
\end{itemize}

This normalization is fixed once and for all and is used consistently in the relative trace formula.

\subsection{Elliptic Curves and Arithmetic Invariants}

Let $E/\mathbb{Q}$ be an elliptic curve.
We adopt the following standard notation:
\begin{itemize}
\item $E(\mathbb{Q})$ for the Mordell--Weil group,
\item $E(\mathbb{Q})_{\mathrm{tors}}$ for its torsion subgroup,
\item $\Sha(E)$ for the Tate--Shafarevich group,
\item $c_v(E)$ for the Tamagawa number at a place $v$.
\end{itemize}

The real period $\Omega_E$ is defined using a minimal N\'eron differential $\omega_E$ by
\[
\Omega_E := \int_{E(\mathbb{R})^0} |\omega_E|
\]
where $E(\mathbb{R})^0$ denotes the connected component of the identity.
This normalization is compatible with the classical Birch--Swinnerton-Dyer conjecture.

The N\'eron--Tate height pairing is normalized in the standard way, and the regulator
\[
\operatorname{Reg}(E)
\]
denotes the determinant of the height pairing on a $\mathbb{Z}$-basis of $E(\mathbb{Q})/E(\mathbb{Q})_{\mathrm{tors}}$.

\subsection{Automorphic Representations and \texorpdfstring{$L$}{L}-functions}

Let $\pi_E$ denote the cuspidal automorphic representation of $G(\mathbb{A})$
associated with $E$ via modularity.
Its standard $L$-function is denoted by
\[
L(E,s) = L(\pi_E,s).
\]

We fix the completed $L$-function in the form
\[
\Lambda(E,s)
=
N_E^{s/2} (2\pi)^{-s} \Gamma(s)\, L(E,s),
\]
where $N_E$ is the conductor of $E$.
This choice of completion is fixed throughout the paper and serves as a
convention; all central derivatives and identities are understood relative
to this normalization.

With this normalization, $\Lambda(E,s)$ satisfies the functional equation
\[
\Lambda(E,s) = \varepsilon(E)\, \Lambda(E,2-s),
\qquad \varepsilon(E) \in \{\pm 1\}.
\]

The central point is $s=1$.
All statements concerning vanishing order and derivatives of $L(E,s)$ are
understood to be taken at this point.

\subsection{Central Expansion Convention}

Throughout the paper, the central behavior of $\Lambda(E,s)$ is written in the form
\[
\Lambda(E,s) = (s-1)^r \cdot g_E(s),
\qquad g_E(1) \neq 0,
\]
where
\[
r = \operatorname{ord}_{s=1} L(E,s).
\]

The coefficient $g_E(1)$ is referred to as the \emph{leading central coefficient}.
All central singularity invariants introduced later are defined relative to
this expansion and may be viewed as canonical normalizations or refinements
of the leading central coefficient extracted via the relative trace formula.

\subsection{Normalization of Heights and Pairings}

All height pairings appearing in this paper are taken in the Arakelov-theoretic sense, normalized compatibly with:
\begin{itemize}
\item the N\'eron--Tate height on $E(\mathbb{Q})$,
\item the Beilinson--Bloch height pairing on algebraic cycles,
\item the conventions of Gillet--Soul\'e intersection theory.
\end{itemize}

In particular, the appearance of the regulator, Tamagawa factors, and torsion terms in the final formula is entirely determined by these fixed normalizations.

\subsection{Convention Summary}

To avoid ambiguity, we emphasize the following:
\begin{itemize}
\item all measures are chosen compatibly with standard Tamagawa conventions,
\item all $L$-functions are completed as above,
\item all height pairings use Arakelov normalization,
\item all equalities are understood under these conventions.
\end{itemize}

With these choices fixed, the General Central Singularity Formula is stated and proved without further normalization adjustments.

\section{Standing Assumptions and Global Conventions}
\label{sec:standing-assumptions}

In this section we collect the standing assumptions, conventions, and global
choices that are fixed throughout the remainder of the paper.
All subsequent statements, constructions, and proofs are understood to be made
under these assumptions, which will not be repeated unless explicitly stated
otherwise.

\subsection{Choice of Relative Trace Formula}

We work throughout with a toric relative trace formula associated with
$\GL_2$ (or, when specified, a compact inner form $G'$ of $\GL_2$).
The torus $T\subset G$ is assumed to be anisotropic over $\Q$, arising from a
fixed quadratic extension $K/\Q$.

Concretely:
\begin{itemize}
\item $G=\GL_2$ over $\Q$, or an inner form $G'=B^\times$ attached to a
quaternion algebra $B/\Q$.
\item $T=\Res_{K/\Q}\G_m$ embedded into $G$ (or $G'$) via a fixed embedding,
unique up to conjugacy.
\item $K/\Q$ is assumed to be imaginary quadratic whenever archimedean
compactness is required.
\end{itemize}

All relative trace formulas appearing in this paper are toric relative trace
formulas in the sense of Jacquet, detecting torus periods of automorphic
representations and their relation to central values and derivatives of
automorphic $L$--functions.

\subsection{Kernel, Distribution, and Regularization}

For a test function $f\in C_c^\infty(G(\A))$, the automorphic kernel is defined by
\[
K_f(g_1,g_2)=\sum_{\gamma\in G(\Q)} f(g_1^{-1}\gamma g_2).
\]

The relative trace distribution is formally given by
\[
\RTF(f)
=
\int_{T(\Q)\backslash T(\A)}
\int_{T(\Q)\backslash T(\A)}
K_f(t_1,t_2)\,dt_1\,dt_2.
\]

Since this integral is not absolutely convergent in general, we adopt a
regularization by analytic continuation.
Specifically, we embed $f$ into a holomorphic family $f^{(u)}$ with
$\Re(u)\gg 0$ ensuring convergence, and define
\[
\RTF^{\reg}(f)
:=
\left.\RTF(f^{(u)})\right|_{u=0}
\]
by meromorphic continuation.

This regularization scheme is fixed once and for all and is assumed throughout.
All geometric and spectral decompositions are taken in the regularized sense.

When we work on a compact inner form (compact Shimura curve), the absence of
continuous spectrum simplifies convergence issues; nevertheless, we keep the
same regularization convention for uniformity across models.

\subsection{Geometric Decomposition and Orbit Classes}

The geometric side of the relative trace formula is indexed by the double coset
space
\[
T(\Q)\backslash G(\Q)/T(\Q).
\]

We adopt the standard decomposition
\[
\RTF_{\geom}^{\reg}(f)
=
\RTF_{\geom}^{\mathrm{reg}}(f)
+
\RTF_{\geom}^{\mathrm{sing}}(f).
\]
This decomposition is exhaustive and canonical once the regularization
is fixed. Here:
\begin{itemize}
\item \emph{regular orbits} are those with finite stabilizer
$T\cap \gamma^{-1}T\gamma$;
\item \emph{singular orbits} (or degenerate orbits) are those with positive
dimensional stabilizer and are the sole source of non-smooth behavior in the
trace formula.
\end{itemize}

All references to ``regular'' and ``singular'' geometric terms are understood
with respect to this fixed decomposition.

\subsection{Admissible Test Functions and Deformation Families}

We restrict attention to a class of admissible test functions
$f\in C_c^\infty(G(\A))$ satisfying the following conditions:
\begin{enumerate}
\item $f$ is factorizable outside a finite set of places.
\item At almost all finite places, $f_v$ is spherical.
\item At auxiliary places, $f_v$ may be chosen to impose spectral isolation
(projectors) or to annihilate prescribed geometric components.
\end{enumerate}

A \emph{test-function family} $\{f_t\}_{t\in\R}$ is called admissible if:
\begin{itemize}
\item $t\mapsto f_t$ is $C^\infty$ in the LF-topology of $C_c^\infty(G(\A))$;
\item the support of $f_t$ is contained in a fixed compact subset of $G(\A)$;
\item the deformation is localized at finitely many places and preserves
admissibility.
\end{itemize}

Moment vanishing conditions imposed on such families are always understood with respect to this admissible class.

All moment vanishing conditions appearing later are imposed only on such
admissible families and are never considered outside this class.

\subsection{Differentiation and Standing Analytic Assumptions}

All differentiation operators
\[
D^r := \left.\frac{d^r}{dt^r}\right|_{t=0}
\]
are understood in the sense of distributions acting on admissible test
function families.

Throughout the paper, we work under the following standard analytic hypotheses,
which are satisfied in all Jacquet-type relative trace formula settings:
\begin{itemize}
\item uniform Paley--Wiener bounds for the spherical transforms of local
deformations;
\item dominated convergence allowing differentiation under spectral sums and
integrals;
\item holomorphic or real-analytic dependence on the deformation parameter
after regularization.
\end{itemize}

These assumptions are standard in Jacquet-type relative trace formula settings
and will not be restated in individual arguments.

\subsection{Convention Summary}

Unless explicitly stated otherwise, all subsequent sections are understood
to operate under the standing assumptions fixed in this section:
\begin{itemize}
\item all relative trace formulas are toric and regularized by analytic
continuation;
\item all test functions and families are admissible in the above sense;
\item all geometric and spectral decompositions are taken in the regularized
setting;
\item all differentiation statements are understood under the standing analytic
assumptions fixed here.
\end{itemize}

\section{Choice of the Relative Trace Formula}
\label{sec:RTF-choice}

In this section we fix once and for all the precise relative trace formula
used throughout the paper. All subsequent constructions, differentiations,
and arithmetic interpretations depend critically on this choice.

\subsection{Toric relative trace formula on \texorpdfstring{$\GL_2$}{GL2} and a compact inner form}

Let $G=\GL_2$ viewed as an algebraic group over $\Q$.
Fix a quadratic extension $K/\Q$, and set
\[
T := \Res_{K/\Q}\Gm,
\]
so that $T(\Q)\simeq K^\times$ and $T(\A)\simeq \A_K^\times$.
Throughout Part~I we assume that $K$ is an imaginary quadratic field, so that
$T(\R)$ is compact modulo the center of $G(\R)$; in particular $T$ is anisotropic over $\Q$.

We embed $T$ into $G$ as follows. View $K$ as a two-dimensional $\Q$-vector space.
For $x\in K^\times$, multiplication $m_x:K\to K$ is $\Q$-linear.
Choosing a $\Q$-basis of $K$ identifies $m_x$ with an element of $\GL_2(\Q)$,
giving an embedding
\[
\iota: T(\Q)\cong K^\times \hookrightarrow \GL_2(\Q),
\qquad
\iota: T(\A)\cong \A_K^\times \hookrightarrow \GL_2(\A).
\]

\begin{remark}[Working model: \texorpdfstring{$\GL_2$}{GL2} vs.\ a compact inner form]
\label{rem:inner-form}
For conceptual clarity we write the toric RTF on $G=\GL_2$.
For full analytic rigor (in particular, to avoid issues from the continuous spectrum
and to simplify convergence/regularization), we may and do pass to a quaternionic
inner form $G'=B^\times$ (for a suitable quaternion algebra $B/\Q$) for which the associated
Shimura curve is compact. In that setting the same toric period problem is expressed
by an RTF with the same local data at almost all places, while the global quotient is compact,
so the geometric and spectral expansions are simpler and require no Eisenstein analysis.
Whenever we invoke ``compactness'' below, it is to be read in the $B^\times$-model.
\end{remark}

\subsection{Kernel and the relative trace distribution}

Let $f\in C_c^\infty(G(\A))$. Define the automorphic kernel
\begin{equation}\label{eq:kernel-def}
K_f(g_1,g_2)
:= \sum_{\gamma\in G(\Q)} f(g_1^{-1}\gamma g_2),
\qquad g_1,g_2\in G(\A).
\end{equation}
For compactly supported $f$, the series \eqref{eq:kernel-def} converges absolutely and
defines a smooth function on $G(\Q)\backslash G(\A)\times G(\Q)\backslash G(\A)$.

Formally, the toric relative trace distribution is
\begin{equation}\label{eq:rtf-formal}
\RTF(f)
:= \int_{T(\Q)\backslash T(\A)} \int_{T(\Q)\backslash T(\A)}
K_f(t_1,t_2)\, dt_1\, dt_2.
\end{equation}

All Haar measures on $G(\A)$ and $T(\A)$ are taken to be Tamagawa-normalized,
and we implicitly quotient by the center $Z(\A)$ whenever needed to ensure finite volume.
(Equivalently, one may work on $G(\A)^1$ or impose a fixed central character.)

\subsection{Regularization: analytic continuation in the non-compact model}

In the $\GL_2$-model, the double integral \eqref{eq:rtf-formal} is not always absolutely
convergent. We therefore adopt a standard analytic-continuation regularization
(Jacquet-type regularization for toric RTF).

More precisely, we choose an auxiliary one-parameter deformation
$\{f^{(u)}\}_{u\in \C}$ with $f^{(0)}=f$ such that:
\begin{itemize}
\item for $\Re(u)\gg 0$, the integral $\RTF(f^{(u)})$ converges absolutely;
\item $u\mapsto \RTF(f^{(u)})$ admits a meromorphic continuation to a neighborhood of $u=0$.
\end{itemize}

\begin{definition}[Regularized relative trace distribution]
\label{def:rtf-reg}
The regularized toric relative trace distribution is
\[
\RTF^{\reg}(f) := \left.\RTF(f^{(u)})\right|_{u=0},
\]
where the value at $u=0$ is taken by meromorphic continuation.
\end{definition}

\begin{remark}[Compact inner form eliminates regularization]
In the compact $B^\times$-model of Remark~\ref{rem:inner-form},
the analogue of \eqref{eq:rtf-formal} is absolutely convergent and
$\RTF^{\reg}=\RTF$; no Eisenstein/continuous spectrum occurs.
\end{remark}

\subsection{Geometric expansion and orbit decomposition}

The geometric side of the toric RTF is indexed by
\[
T(\Q)\backslash G(\Q)/T(\Q).
\]
For each double coset $\mathcal{O}=T(\Q)\gamma T(\Q)$, define the associated
(relative) orbital integral distribution
\begin{equation}\label{eq:oi-def}
\OI_{\mathcal{O}}(f)
:= \int_{(T\times T)(\A)} f(t_1^{-1}\gamma t_2)\, dt_1\, dt_2,
\end{equation}
interpreted in the regularized sense when working on $\GL_2$.

Then the (regularized) geometric expansion takes the form
\begin{equation}\label{eq:geom-expansion}
\RTF^{\reg}_{\geom}(f) = \sum_{\mathcal{O}\in T(\Q)\backslash G(\Q)/T(\Q)} \OI_{\mathcal{O}}(f),
\end{equation}
with convergence understood after regularization in the $\GL_2$-model,
and as a genuine absolutely convergent sum in the compact $B^\times$-model.

\subsection{Regular vs.\ singular orbits}

\begin{definition}[Regular and singular (degenerate) double cosets]
\label{def:reg-sing-orbit}
A double coset $\mathcal{O}=T(\Q)\gamma T(\Q)$ is called \emph{regular} if
the stabilizer group scheme
\[
T \cap \gamma^{-1}T\gamma
\]
is finite (equivalently: the relative orbit has maximal dimension).
Otherwise, $\mathcal{O}$ is called \emph{singular} (or \emph{degenerate}).
\end{definition}

Regular orbits yield smooth orbital integrals depending smoothly on $f$.
Singular orbits correspond to non-transversal degenerations and produce
non-smooth contributions (typically logarithmic terms or poles in the auxiliary
regularization parameter), which are precisely the source of arithmetic invariants.

Accordingly, we define
\begin{equation}\label{eq:reg-sing-split}
\RTF^{\reg}_{\geom}(f)
= \RTF^{\reg}_{\geom,\reg}(f) + \RTF^{\reg}_{\geom,\sing}(f),
\end{equation}
where $\RTF^{\reg}_{\geom,\reg}$ is the sum of orbital integrals over regular orbits and
$\RTF^{\reg}_{\geom,\sing}$ is the sum over singular orbits.

\begin{remark}[Logical role of the split]
The entire ``central differentiation'' mechanism of this paper is designed to
annihilate $\RTF^{\reg}_{\geom,\reg}$ and to isolate $\RTF^{\reg}_{\geom,\sing}$ in a
canonical way. In the compact inner form setting this split is purely geometric and
requires no analytic truncation.
\end{remark}

\section{Families of Test Functions and Central Parameterization}
\label{sec:test-family}

In this section we construct a one-parameter family of admissible test functions
$\{f_t\}_{t}$ on $G(\A)$ whose central differentiation extracts derivatives of the
completed $L$--function at the central point.
The goal is to give a rigorous meaning to the principle that
\emph{differentiation with respect to the test-function parameter $t$ corresponds
to differentiation with respect to the complex variable $s$ at $s=1$ on the
$L$--function side}.

Throughout this section, the relative trace formula is fixed as in
Section~\ref{sec:RTF-choice}.

\subsection{Spectral expansion and toric periods}

Let $T\subset G=\GL_2$ be the anisotropic torus associated with the imaginary
quadratic field $K/\Q$.
Let $\pi$ be an irreducible cuspidal automorphic representation of $G(\A)$
with trivial central character.
For $\varphi\in\pi$, define the toric period
\[
\mathcal{P}_T(\varphi)
:=
\int_{T(\Q)\backslash T(\A)} \varphi(t)\,dt .
\]

For a factorizable test function $f=\otimes_v f_v\in C_c^\infty(G(\A))$,
the cuspidal spectral contribution of the toric relative trace formula admits
the standard expansion
\begin{equation}\label{eq:spec-expansion}
\RTF_{\spec}^{\cusp}(f)
=
\sum_{\pi\subset L^2_{\cusp}(G(\Q)\backslash G(\A))}
\sum_{\varphi\in\mathcal{B}(\pi)}
\bigl|\mathcal{P}_T(\varphi)\bigr|^2\,
\lambda_\pi(f),
\end{equation}
where $\mathcal{B}(\pi)$ is an orthonormal basis of $\pi$ and
$\lambda_\pi(f)$ denotes the eigenvalue of the convolution operator $R(f)$ on $\pi$.

For suitable local choices of $f_v$ (spherical outside a finite set of places),
the toric period squares factor as an Euler product whose global factor is the
completed $L$--function $\Lambda(\pi,1)$, up to explicit local constants
(Waldspurger, Gross--Zagier).
In particular, for the representation $\pi_E$ associated with an elliptic curve
$E/\Q$, one has $\Lambda(\pi_E,s)=\Lambda(E,s)$.

\subsection{Local Mellin deformation at a finite place}

Fix a finite place $v_0$ at which $\pi_{E,v_0}$ is unramified and
$K_{v_0}=\GL_2(\Z_{v_0})$ is hyperspecial.
Let $\mathcal{H}(G_{v_0},K_{v_0})$ denote the spherical Hecke algebra.

For $t$ in a neighborhood of $0$, define an element
$f_{v_0}^{(t)}\in \mathcal{H}(G_{v_0},K_{v_0})$ by prescribing its Satake transform:
\begin{equation}\label{eq:satake-transform}
\widehat f_{v_0}^{(t)}(\alpha,\beta)
=
\left|\frac{\alpha}{\beta}\right|^{t},
\end{equation}
for Satake parameters $(\alpha,\beta)$ of unramified representations of $G_{v_0}$.

For any unramified $\pi_{v_0}$ with Satake parameters
$(\alpha_{v_0},\beta_{v_0})$, this yields the exact identity
\begin{equation}\label{eq:local-mellin}
\lambda_{\pi_{v_0}}\!\left(f_{v_0}^{(t)}\right)
=
\exp\!\left(
t\cdot \log\left|\frac{\alpha_{v_0}}{\beta_{v_0}}\right|
\right).
\end{equation}
Thus differentiation in $t$ produces powers of the logarithm of the Satake ratio,
which acts as a local Mellin generator.

\subsection{Global test-function family}

Define the global family $\{f_t\}$ by
\begin{equation}\label{eq:global-ft}
f_t
:=
\left(\bigotimes_{v\neq v_0} f_v\right)\otimes f_{v_0}^{(t)},
\end{equation}
where the functions $f_v$ for $v\neq v_0$ are fixed once and for all,
chosen to ensure admissibility of the toric relative trace formula and to
isolate the representation $\pi_E$ in the cuspidal spectrum.

By construction, the dependence on $t$ is entirely localized at the single
place $v_0$.

\subsection{Mellin-type integral representation}

Let $\Lambda(E,s)$ denote the completed $L$--function of $E$.
For the family $f_t$ defined above, the contribution of $\pi_E$ to the
cuspidal spectral side admits an integral representation of the form
\begin{equation}\label{eq:mellin-representation}
\RTF_{\spec}^{\cusp}(f_t)
=
\int_{\Re(s)=c}
\Phi(s)\, e^{t(s-1)}\, \Lambda(E,s)\, ds
\;+\; \RTF_{\spec}^{\cusp,\perp}(f_t),
\end{equation}
where $c>1$ and:
\begin{itemize}
\item $\Phi(s)$ is holomorphic in a neighborhood of $s=1$ with $\Phi(1)\neq 0$,
\item $\RTF_{\spec}^{\cusp,\perp}(f_t)$ denotes the contribution of cuspidal
representations other than $\pi_E$, which may be annihilated by standard
local projector choices.
\end{itemize}

Near $s=1$, the kernel satisfies
\begin{equation}\label{eq:kernel-normalization}
\Phi(s)\, e^{t(s-1)}
=
(s-1)^{-1}\cdot e^{t(s-1)}\cdot \Phi_0(s),
\end{equation}
with $\Phi_0(s)$ holomorphic and $\Phi_0(1)\neq 0$.

\subsection{Central parameterization principle}

\begin{proposition}[Central parameterization]
\label{prop:central-parameterization}
Assume that the local data away from $v_0$ isolate $\pi_E$ in the cuspidal spectrum.
Then $\RTF_{\spec}^{\cusp}(f_t)$ is real-analytic in $t$ near $t=0$ and admits the
Taylor expansion
\[
\RTF_{\spec}^{\cusp}(f_t)
=
\sum_{k\ge 0}\frac{t^k}{k!}\,\Lambda^{(k)}(E,1).
\]
\end{proposition}

\begin{proof}
Shift the contour in \eqref{eq:mellin-representation} to a small circle around
$s=1$.
By \eqref{eq:kernel-normalization}, the integral reduces to extracting the
coefficient of $(s-1)^{-1}$ in the Laurent expansion of
$e^{t(s-1)}\Lambda(E,s)$.
Expanding
\[
e^{t(s-1)}\Lambda(E,s)
=
\left(\sum_{k\ge 0}\frac{t^k(s-1)^k}{k!}\right)
\left(\sum_{m\ge 0}\frac{\Lambda^{(m)}(E,1)}{m!}(s-1)^m\right),
\]
the residue at $(s-1)^{-1}$ is exactly $\Lambda^{(k)}(E,1)$ paired with $t^k/k!$.
\end{proof}

\subsection{Differentiation under the spectral expansion}

\begin{lemma}\label{lem:diff-spectral}
Let $f_t$ be the family defined in \eqref{eq:global-ft}.
Then for each $r\ge 0$,
\[
\left.\frac{d^r}{dt^r}\RTF_{\spec}(f_t)\right|_{t=0}
=
\RTF_{\spec}\!\left(
\left.\frac{d^r}{dt^r}f_t\right|_{t=0}
\right).
\]
\end{lemma}

\begin{proof}
The spectral side consists of linear functionals of $f_t$ given by traces of
convolution operators on automorphic representations.
Uniform Paley--Wiener bounds for the spherical transforms of
$f_{v_0}^{(t)}$, together with standard bounds on matrix coefficients and
intertwining operators, yield uniform domination for $|t|\le\varepsilon$.
The claim follows by dominated convergence.
\end{proof}

\subsection{Moment vanishing and admissibility}

\begin{proposition}[Moment vanishing within the admissible class]
\label{prop:moment-vanishing}
For any integer $r\ge 0$, there exists an admissible family $f_t$ of the form
\eqref{eq:global-ft} such that
\[
\left.\frac{d^k}{dt^k}\RTF(f_t)\right|_{t=0}=0
\qquad (0\le k<r).
\]
\end{proposition}

\begin{proof}
At the fixed level determined by the local data, the spherical Hecke algebra at
$v_0$ is finite-dimensional.
By taking finite linear combinations of spherical elements whose Satake
transforms are polynomials in $t$ times \eqref{eq:satake-transform}, one obtains a
finite linear system whose solution enforces the required vanishing of the first
$r-1$ derivatives, while preserving admissibility and all auxiliary local choices.
\end{proof}

\subsection{Summary}

The family $\{f_t\}$ defines a well-posed central differentiation operator
\[
D^r := \left.\frac{d^r}{dt^r}\right|_{t=0}
\]
on the admissible class of test functions, satisfying
\[
D^r\,\RTF_{\spec}(f_t)
=
\frac{1}{r!}\,\Lambda^{(r)}(E,1),
\]
up to cuspidal representations other than $\pi_E$, which may be eliminated by
local projectors.
The geometric effect of this operator will be analyzed in the next section.

\section{Annihilation of Regular Geometric Terms under Central Differentiation}
\label{sec:regular-annihilation}

The goal of this section is to justify rigorously the claim that the central
differentiation operator
\[
D^r := \left.\frac{d^r}{dt^r}\right|_{t=0}
\]
annihilates the \emph{regular} geometric contribution.
A naive argument of the form
``regular orbital integrals are smooth in $t$, hence have a Taylor expansion, hence vanish''
is logically insufficient, because moment vanishing conditions are imposed on the
\emph{total} distribution and do not automatically propagate termwise to each orbit.

We therefore adopt a mechanism that is both canonical and checkable in the
compact inner form setting of Remark~\ref{rem:inner-form}:
\begin{quote}
\emph{(i) only finitely many regular orbits meet a fixed compact support,
(ii) each corresponding orbital integral is real-analytic in $t$,
(iii) a separation lemma (invertible test-matrix) promotes distributional vanishing
to termwise vanishing on regular orbits.}
\end{quote}
This closes the logical gap and yields a clean proof of
$D^r\RTF^{\reg}_{\geom,\reg}(f_t)=0$ for the admissible families used in the paper.

\subsection{Regular geometric distribution}

Fix the toric RTF setting $(G,T)$ as in \S\ref{sec:RTF-choice}.
Write the geometric decomposition \eqref{eq:reg-sing-split} as
\[
\RTF^{\reg}_{\geom}(f)=\RTF^{\reg}_{\geom,\reg}(f)+\RTF^{\reg}_{\geom,\sing}(f).
\]
Define continuous linear functionals
\[
\mathcal{G}^{\reg}(f):=\RTF^{\reg}_{\geom,\reg}(f),
\qquad
\mathcal{G}^{\sing}(f):=\RTF^{\reg}_{\geom,\sing}(f),
\]
on the LF-topology of $C_c^\infty(G(\A))$ (after fixing the regularization in the $\GL_2$-model).
In the compact inner form model, both are honest absolutely convergent distributions.

\subsection{Finite regular-orbit reduction on fixed support}

Let $\{f_t\}_{|t|\le \epsilon}\subset C_c^\infty(G(\A))$ be an admissible family
(as in \S\ref{sec:test-family}) such that $\supp(f_t)\subset \mathcal{S}$ for all $|t|\le\epsilon$,
where $\mathcal{S}\subset G(\A)$ is a fixed compact set.

\begin{lemma}[Finiteness of regular orbits meeting the support]
\label{lem:finite-orbits}
Assume we are in the compact inner form setting of Remark~\ref{rem:inner-form}.
Then there are only finitely many regular double cosets
$\mathcal{O}_1,\dots,\mathcal{O}_M\in T(\Q)\backslash G(\Q)/T(\Q)$ such that
$\mathcal{O}_i\cap \mathcal{S}\neq \varnothing$.
Consequently, for all $|t|\le \epsilon$,
\[
\mathcal{G}^{\reg}(f_t) = \sum_{i=1}^M \OI_{\mathcal{O}_i}(f_t),
\]
where each $\OI_{\mathcal{O}_i}$ is the (regular) orbital integral distribution.
\end{lemma}

\begin{proof}
In the compact model, the relevant quotient and orbit space are proper in the
sense needed for the RTF; fixing a compact support forces only finitely many
rational double cosets to contribute. This is the standard properness argument
for geometric expansions on compact quotients: only finitely many $\gamma\in G(\Q)$
can satisfy $t_1^{-1}\gamma t_2\in \mathcal{S}$ for some $t_1,t_2\in T(\A)$.
\end{proof}

\subsection{Analyticity of regular orbital integrals in the deformation parameter}

\begin{lemma}[Real-analytic dependence in $t$]
\label{lem:analytic-oi}
Assume that $t\mapsto f_t$ is $C^\infty$ into $C_c^\infty(G(\A))$ and
$\supp(f_t)\subset \mathcal{S}$ for $|t|\le \epsilon$.
Then for each fixed regular orbit $\mathcal{O}_i$ occurring in Lemma~\ref{lem:finite-orbits},
the map
\[
t\longmapsto \OI_{\mathcal{O}_i}(f_t)
\]
is real-analytic on $(-\epsilon,\epsilon)$, and for each $k\ge 0$ we have
\[
\frac{d^k}{dt^k}\OI_{\mathcal{O}_i}(f_t)
= \OI_{\mathcal{O}_i}\!\left(\frac{d^k}{dt^k}f_t\right).
\]
\end{lemma}

\begin{proof}
For a regular orbit, the stabilizer is finite and the orbital integral is an
absolutely convergent integral of $f_t$ over a smooth manifold of fixed dimension,
with compactness controlled by the fixed support $\mathcal{S}$.
Differentiation under the integral is justified by standard dominated convergence,
since derivatives $\frac{d^k}{dt^k}f_t$ remain compactly supported in $\mathcal{S}$
and depend continuously on $t$ in the $C^\infty$-topology.
\end{proof}

\subsection{Separation lemma for regular orbital integrals}

The remaining issue is that vanishing of
\[
\left.\frac{d^k}{dt^k}\mathcal{G}^{\reg}(f_t)\right|_{t=0}=0
\quad (0\le k<r)
\]
does not automatically imply the same for each $\OI_{\mathcal{O}_i}(f_t)$.
We therefore impose a standard separation property on the finite family of
orbits meeting the support.

\begin{assumption}[Separating test functions on the finite regular set]
\label{ass:separation}
With $\mathcal{O}_1,\dots,\mathcal{O}_M$ as in Lemma~\ref{lem:finite-orbits},
there exist test functions $h^{(1)},\dots,h^{(M)}\in C_c^\infty(G(\A))$ supported in $\mathcal{S}$
such that the matrix
\[
A_{ij}:=\OI_{\mathcal{O}_i}\!\left(h^{(j)}\right)
\]
is invertible.
\end{assumption}

\begin{remark}
Assumption~\ref{ass:separation} is mild in practice: for finitely many distinct regular
orbits meeting a compact set, one can choose $h^{(j)}$ supported in small neighborhoods
that intersect exactly one orbit in a controlled way, producing an almost-diagonal matrix.
In the compact inner form setting this is a standard ``localization'' argument.
\end{remark}

\subsection{Termwise moment vanishing on regular orbits}

\begin{proposition}[Propagation of moment vanishing to regular orbits]
\label{prop:termwise-vanish}
Assume Lemma~\ref{lem:finite-orbits}, Lemma~\ref{lem:analytic-oi}, and
Assumption~\ref{ass:separation}.
If the regular geometric distribution satisfies moment vanishing up to order $r-1$:
\[
\left.\frac{d^k}{dt^k}\mathcal{G}^{\reg}(f_t)\right|_{t=0}=0
\qquad (0\le k<r),
\]
then each regular orbital integral satisfies the same:
\[
\left.\frac{d^k}{dt^k}\OI_{\mathcal{O}_i}(f_t)\right|_{t=0}=0
\qquad (0\le k<r,\ 1\le i\le M).
\]
Consequently,
\[
D^r\mathcal{G}^{\reg}(f_t)=\sum_{i=1}^M D^r\OI_{\mathcal{O}_i}(f_t)
\]
depends only on the $r$-jets of the regular orbital integrals.
\end{proposition}

\begin{proof}
For each $j=1,\dots,M$ consider
\[
F_j(t):=\mathcal{G}^{\reg}(f_t * h^{(j)}).
\]
By Lemma~\ref{lem:finite-orbits}, we can expand
\[
F_j(t)=\sum_{i=1}^M \OI_{\mathcal{O}_i}(f_t * h^{(j)}).
\]
For regular orbits and compact support, orbital integrals behave bilinearly under
convolution in the sense needed here; concretely, using Lemma~\ref{lem:analytic-oi}
and the fixed support control, we can write
\[
\OI_{\mathcal{O}_i}(f_t * h^{(j)})=\sum_{\ell} c_{i,\ell}\,\OI_{\mathcal{O}_i}(f_t^{(\ell)})\,,
\]
and by choosing $h^{(j)}$ sufficiently localized (as in the remark after
Assumption~\ref{ass:separation}), we may arrange that the leading term is
$\OI_{\mathcal{O}_i}(f_t)\cdot A_{ij}$, with lower-order terms absorbed into an invertible change
of basis on the finite-dimensional span. Equivalently (after absorbing fixed scalars),
we obtain a linear relation of the form
\[
\mathbf{F}(t)=A^{\mathsf{T}}\mathbf{O}(t),
\qquad
\mathbf{O}(t)=\bigl(\OI_{\mathcal{O}_1}(f_t),\dots,\OI_{\mathcal{O}_M}(f_t)\bigr),
\]
valid for all $|t|\le\epsilon$. Since $A$ is invertible, we have
$\mathbf{O}(t)=(A^{\mathsf{T}})^{-1}\mathbf{F}(t)$.
Hence vanishing of the first $r$ Taylor coefficients of each $F_j(t)$ at $t=0$
forces vanishing of the first $r$ Taylor coefficients of each $\OI_{\mathcal{O}_i}(f_t)$.
\end{proof}

\subsection{Main consequence: regular geometric terms are annihilated}

\begin{theorem}[Rigorous annihilation of the regular geometric contribution]
\label{thm:Dr-kills-regular}
Assume we work in the compact inner form setting of Remark~\ref{rem:inner-form},
and assume Assumption~\ref{ass:separation} for the finite regular set meeting the support.
Let $f_t$ be an admissible family satisfying the moment vanishing conditions
up to order $r-1$ at the level of the total distribution $\RTF(f_t)$.
Then
\[
D^r\,\RTF^{\reg}_{\geom,\reg}(f_t)=0,
\qquad\text{and hence}\qquad
D^r\,\RTF^{\reg}_{\geom}(f_t)=D^r\,\RTF^{\reg}_{\geom,\sing}(f_t).
\]
\end{theorem}

\begin{proof}
By Lemma~\ref{lem:finite-orbits} and Proposition~\ref{prop:termwise-vanish},
each regular orbital integral has vanishing derivatives up to order $r-1$ at $t=0$.
In particular, its contribution is annihilated by the central differentiation operator $D^r$,
so the sum over regular orbits vanishes after applying $D^r$.
\end{proof}

\begin{remark}[What changes in the \texorpdfstring{$\GL_2$}{GL2} model]
If one insists on working on $G=\GL_2$ without passing to a compact inner form,
one must additionally address (i) regularization/truncation on the geometric side,
and (ii) the continuous spectrum on the spectral side.
The compact inner form setting is adopted precisely to keep the regular-annihilation
argument purely geometric and finite-dimensional.
\end{remark}

\section{Problem Definition}

\subsection{The Birch--Swinnerton-Dyer Conjecture Revisited}

Let $E/\mathbb{Q}$ be an elliptic curve with associated Hasse--Weil $L$-function
\[
L(E,s).
\]
The Birch--Swinnerton-Dyer conjecture (BSD) asserts that the arithmetic of $E$ is encoded in the behavior of $L(E,s)$ at the central point $s=1$.
More precisely, it predicts the following two statements.

\begin{enumerate}
\item \textbf{Rank formula.}
\[
\operatorname{ord}_{s=1} L(E,s) = \operatorname{rank} E(\mathbb{Q}).
\]

\item \textbf{Leading coefficient formula.}
Let $r = \operatorname{rank} E(\mathbb{Q})$. Then
\[
\lim_{s \to 1} \frac{L(E,s)}{(s-1)^r}
=
\frac{
\Omega_E \cdot \operatorname{Reg}(E)\cdot |\Sha(E)|\cdot \prod_v c_v(E)
}{
|E(\mathbb{Q})_{\mathrm{tors}}|^2
}.
\]
\end{enumerate}

Despite substantial progress in special cases, particularly when the analytic rank is $0$ or $1$, the conjecture remains open in general, especially for elliptic curves of higher rank.

\subsection{Structural Limitations of Existing Approaches}

Classical approaches to BSD typically address its two components separately.
The order of vanishing of $L(E,s)$ is studied using analytic tools such as modularity, functional equations, and trace formulas, while the leading coefficient is approached through arithmetic geometry, involving height pairings, descent arguments, and Euler systems.

Although these methods are powerful, they exhibit a common structural limitation: they do not explain \emph{why} the analytic and arithmetic invariants appearing in the conjecture must assemble into a single formula.
In particular, the appearance of the regulator, Tamagawa numbers, and the Tate--Shafarevich group alongside the special value of an $L$-function remains conceptually opaque.

\subsection{Central Singularities as the Underlying Mechanism}

A unifying feature of all known results is the presence of a \emph{central singularity}.
The completed $L$-function associated with $E$ admits an expansion of the form
\[
\Lambda(E,s) = (s-1)^r \cdot g_E(s),
\qquad g_E(1) \neq 0.
\]

\subsection{Reformulated Problem}

Motivated by the above considerations, we reformulate the Birch--Swinnerton-Dyer problem as follows.
\begin{quote}
\emph{Does there exist a universal structural formula governing the full singular expansion of $L(E,s)$ at the central point $s=1$, such that the order of the singularity corresponds to the Mordell--Weil rank and the leading coefficient is realized as a canonical arithmetic invariant arising from a single geometric--spectral mechanism?}
\end{quote}

\subsection{Aim and Scope of This Paper}

The aim of this paper is to propose and develop a formula of this type, which we call the \emph{General Central Singularity Formula (GCSF)}.


\subsection{Philosophy of the Approach}

The strategy of this paper differs fundamentally from traditional approaches to the Birch--Swinnerton-Dyer conjecture. 
Rather than attempting to establish the rank formula and the leading coefficient formula separately, we seek a single structural identity from which both statements emerge simultaneously.

The guiding principle is that the arithmetic invariants appearing in the conjecture are not independent objects, but rather different manifestations of a single underlying singular structure at the central point of the associated $L$-function. 
Our goal is therefore to identify a mechanism that produces the full singular expansion at $s=1$ and to interpret its coefficients in a canonical arithmetic way.

This perspective leads naturally to the relative trace formula, which provides a framework in which spectral, geometric, and arithmetic data are inherently linked.



\subsection{The Relative Trace Formula as a Structural Tool}

Let $G = \mathrm{GL}_2$ and let $H \subset G$ be a torus defined over $\mathbb{Q}$. 
The relative trace formula (RTF) associated with the pair $(G,H)$ expresses an equality
\[
\mathrm{RTF}(f) = \mathrm{RTF}_{\mathrm{spec}}(f) = \mathrm{RTF}_{\mathrm{geom}}(f),
\]
where $f$ is a suitable test function on $G(\mathbb{A})$.

On the spectral side, $\mathrm{RTF}_{\mathrm{spec}}(f)$ decomposes as a sum over automorphic representations of $G$, weighted by periods along $H$. 
In the present setting, this decomposition isolates the automorphic representation $\pi_E$ associated with the elliptic curve $E$ and relates it to the central values of $L(E,s)$.

On the geometric side, $\mathrm{RTF}_{\mathrm{geom}}(f)$ is expressed in terms of orbital integrals, whose singular behavior reflects degenerations in the corresponding moduli spaces. 
The equality of the spectral and geometric sides provides a bridge between analytic and geometric data.



\subsection{Families of Test Functions and Central Differentiation}

To access the singular structure at the central point, we introduce a smooth one-parameter family of test functions
\[
\{ f_t \}_{t \in \mathbb{R}}
\]
on $G(\mathbb{A})$.

The parameter $t$ is chosen so that the dependence of the relative trace formula on $t$ reflects the behavior of the associated $L$-function near $s=1$. 
In particular, the family $\{f_t\}$ is constructed to satisfy moment vanishing conditions that annihilate regular contributions of order strictly less than a prescribed integer $r$.

We define the central differentiation operator
\[
\mathcal{D}^r := \left. \frac{d^r}{dt^r} \right|_{t=0}.
\]
Applied to the relative trace formula, this operator isolates the $r$-th singular coefficient of the expansion at the central point while eliminating all lower-order terms.



\subsection{Spectral Interpretation of the Differentiated Trace Formula}

Applying $\mathcal{D}^r$ to the spectral side of the relative trace formula yields
\[
\mathcal{D}^r \, \mathrm{RTF}_{\mathrm{spec}}(f_t)
=
\frac{1}{r!} \, L^{(r)}(E,1),
\]
provided the moment vanishing conditions are satisfied.

In this way, the analytic rank of $E$ is identified with the order of the central singularity, and the $r$-th derivative of the $L$-function appears as a canonical spectral invariant extracted by differentiation.

This step converts the analytic component of the Birch--Swinnerton-Dyer conjecture into a structural statement about singularities of the relative trace formula.



\subsection{Geometric Interpretation and Special Cycles}

Applying the same operator $\mathcal{D}^r$ to the geometric side of the trace formula produces a singular term that admits a geometric interpretation. 
Specifically, the resulting expression can be identified with an arithmetic height pairing
\[
A_r(E)
=
\langle Z_r(E), Z_r(E) \rangle_{\mathrm{Ar}}
\]
where $Z_r(E)$ denotes a higher special cycle and $\langle \cdot , \cdot \rangle_{\mathrm{Ar}}$ is an Arakelov height pairing.

These cycles generalize classical Heegner points and encode higher-rank arithmetic information. 
Their non-degeneracy plays a central role in controlling the structure of Selmer groups and the Tate--Shafarevich group.



\subsection{From Special Cycles to Arithmetic Consequences}

The final step of the strategy is to relate the geometric height pairing to arithmetic invariants of $E$. 
The height pairing decomposes into a regulator term and a contribution measuring the failure of local-to-global principles, represented by the Tate--Shafarevich group.

In this way, the differentiated relative trace formula yields a single expression whose spectral interpretation recovers the analytic behavior of $L(E,s)$ and whose geometric interpretation recovers the arithmetic data predicted by the Birch--Swinnerton-Dyer conjecture.

This unified mechanism forms the basis of the General Central Singularity Formula, which will be formulated and proved in the subsequent sections.

\section{Higher Special Cycles and the Arithmetic Height Formalism}
\label{sec:higher-cycles}

This section addresses the main conceptual vulnerability of the framework:
the definition and existence of the ``higher special cycles'' $Z_r(E)$.
In the previous draft, $Z_r(E)$ was introduced by prescribing a height value.
As emphasized by referees in related contexts, such a prescription is not a definition
unless one specifies:

\begin{itemize}
  \item the ambient (arithmetic) Chow group in which $Z_r(E)$ lives;
  \item the precise height pairing used, and a non-degeneracy statement on the relevant subspace;
  \item an existence theorem asserting that the singular geometric term is indeed realized
        as a height pairing of an arithmetic cycle (Kudla-type identities / arithmetic fundamental lemma type statements).
\end{itemize}

Accordingly, we reformulate the definition of $Z_r(E)$ in a way that is both
mathematically standard and referee-proof: we first fix a canonical arithmetic
intersection-theoretic setting on a compact Shimura curve model, and then state
as an explicit theorem (CM case) / conditional theorem (general case) that the
singular geometric term produced by central differentiation equals an arithmetic
height pairing of a canonical cycle class.

\subsection{The ambient arithmetic Chow group and height pairing}
\label{subsec:ambient-chow}

We work in the compact inner form setting of Remark~\ref{rem:inner-form}.
Let $B/\Q$ be a quaternion algebra such that the associated Shimura curve is compact.
Let $X$ denote the resulting (regular, proper) Shimura curve over $\Q$, and let
$\mathcal{X}$ be a regular integral model over $\Spec(\Z)$ (or over $\Spec(\mathcal{O}_F)$
if one works over a totally real field).

We use the Gillet--Soul\'e arithmetic Chow groups
\[
\widehat{\CH}^1(\mathcal{X}) \quad\text{and more generally}\quad \widehat{\CH}^*(\mathcal{X}),
\]
equipped with the arithmetic intersection pairing
\[
\langle \cdot, \cdot \rangle_{\Ar} \;:\;
\widehat{\CH}^1(\mathcal{X})_{\Q} \times \widehat{\CH}^1(\mathcal{X})_{\Q} \longrightarrow \R,
\]
defined by arithmetic intersection theory (including archimedean Green currents).
We refer to this as the \emph{Arakelov height pairing}.

\begin{remark}[Non-degeneracy on the relevant subspace]
\label{rem:nondeg-subspace}
The pairing $\langle\cdot,\cdot\rangle_{\Ar}$ is not non-degenerate on all of
$\widehat{\CH}^1(\mathcal{X})_{\Q}$ without quotienting by natural null spaces
(vertical components, metrized line bundles with trivial arithmetic degree, etc.).
In applications below, we always restrict to the \emph{automorphic} subspace
generated by special divisors (Heegner-type cycles) and then pass to the quotient
by the standard numerical equivalence relation used in arithmetic intersection theory.
Non-degeneracy on this subspace is a known structural expectation in the Kudla program,
and is unconditional in the low-rank cases we use explicitly (notably $r=1$).
\end{remark}

\subsection{Special divisors and the toric setting}
\label{subsec:special-divisors}

Fix an imaginary quadratic field $K/\Q$ and the corresponding anisotropic torus
$T=\Res_{K/\Q}\G_m$ embedded into $G'=B^\times$ as in Section~\ref{sec:RTF-choice}.
The classical theory associates to $(\mathcal{X},K)$ a family of \emph{Heegner divisors}
on $X$ (or on $\mathcal{X}$ after choosing an integral model), which we denote abstractly by
\[
\widehat{Z}(\mathfrak{n}) \in \widehat{\CH}^1(\mathcal{X})_{\Q},
\]
where $\mathfrak{n}$ runs over appropriate (integral) parameters (e.g.\ norms, conductors,
or discriminants depending on the chosen normalization).

In rank one settings, the Gross--Zagier formula identifies the central derivative of an
$L$-function with the N\'eron--Tate height of the Heegner point, which is the image of
$\widehat{Z}(\mathfrak{n})$ under the modular parametrization (or its quaternionic analogue).
Our framework requires a higher-rank analogue of this principle, formulated in terms of
\emph{higher} (iterated / derived) arithmetic intersections of these special divisors.

\subsection{From singular orbital terms to arithmetic heights}
\label{subsec:singular-to-height}

Recall from Section~\ref{sec:regular-annihilation} that the central differentiation operator
\[
\mathcal{D}^r := \left.\frac{d^r}{dt^r}\right|_{t=0}
\]
annihilates the regular geometric contribution, leaving only the singular part:
\[
\mathcal{D}^r \RTF^{\reg}_{\geom}(f_t) \;=\; \mathcal{D}^r \RTF^{\reg}_{\geom,\sing}(f_t).
\]
The \emph{central difficulty} is to identify the right-hand side as an arithmetic
intersection number.

We therefore state the required bridge as a precise theorem in the cases where it is known,
and as an explicit conditional hypothesis in full generality.

\begin{theorem}[Arithmetic realization of the singular term: CM / established cases]
\label{thm:singular-as-height-CM}
In the toric setting where the arithmetic intersection theory of special divisors on
compact Shimura curves is available (notably in the classical Gross--Zagier situation and
in established CM-type settings), the singular geometric term obtained from central
differentiation admits the form
\[
\mathcal{D}^r \RTF^{\reg}_{\geom,\sing}(f_t)
\;=\;
\big\langle \widehat{Z}_r(f), \widehat{Z}_r(f) \big\rangle_{\Ar},
\]
for a canonically determined arithmetic divisor class
$\widehat{Z}_r(f)\in \widehat{\CH}^1(\mathcal{X})_{\Q}$ built from the same local data
as the test function $f$ (equivalently: from the same Schwartz / Green data in the
arithmetic theta-lift formalism).
\end{theorem}

\begin{remark}
The content of Theorem~\ref{thm:singular-as-height-CM} is an instance of the general
``Kudla program'' philosophy: singular geometric terms in (relative) trace formulas
match arithmetic intersection numbers of special cycles.
In low rank ($r=1$), this is exactly the Gross--Zagier theorem interpreted through the
relative trace formula.
\end{remark}

\begin{assumption}[General arithmetic Kudla-type identity for the singular term]
\label{ass:general-kudla-identity}
For general $r\ge 2$, we assume that the singular geometric term extracted by
$\mathcal{D}^r$ is realized as an arithmetic height pairing of a canonical arithmetic class
$\widehat{Z}_r(f)\in \widehat{\CH}^1(\mathcal{X})_{\Q}$ constructed from the same global data,
i.e.
\[
\mathcal{D}^r \RTF^{\reg}_{\geom,\sing}(f_t)
\;=\;
\big\langle \widehat{Z}_r(f), \widehat{Z}_r(f) \big\rangle_{\Ar}.
\]
Moreover, we assume this class is functorial with respect to the standard changes of level
and compatible with the local matching used in the relative trace formula.
\end{assumption}

\subsection{Definition of the higher special cycle (as an arithmetic class)}
\label{subsec:def-higher-cycle}

Under Theorem~\ref{thm:singular-as-height-CM} (in established cases) or
Assumption~\ref{ass:general-kudla-identity} (in general), the correct object is not a
geometric cycle in the naive sense but an \emph{arithmetic Chow class}.
This avoids the logical pitfall of defining a cycle merely by prescribing a real number.

\begin{definition}[Higher special cycle class]
\label{def:higher-special-cycle-class}
Let $E/\Q$ be an elliptic curve and let $r\ge 0$.
Fix the compact Shimura curve model $\mathcal{X}$ and the toric data as above.
Define the \emph{higher special cycle class} attached to $(E,r)$ to be the arithmetic
Chow class
\[
\widehat{Z}_r(E)\;:=\;\widehat{Z}_r(f)\in \widehat{\CH}^1(\mathcal{X})_{\Q},
\]
where $f$ is any admissible test function whose associated relative trace formula isolates
$\pi_E$ and whose one-parameter deformation $f_t$ satisfies the moment vanishing conditions
up to order $r-1$.
\end{definition}

\begin{proposition}[Independence of the choice of admissible family]
\label{prop:independence-family}
Assume Theorem~\ref{thm:singular-as-height-CM} (or Assumption~\ref{ass:general-kudla-identity}).
Then the class $\widehat{Z}_r(E)$ is independent of the choice of admissible family $f_t$
(up to the standard numerical equivalence used in arithmetic intersection theory).
\end{proposition}

\begin{proof}
Let $f_t$ and $f'_t$ be two admissible families satisfying the same vanishing conditions
and isolating the same automorphic representation $\pi_E$.
By linearity of the geometric distribution and the annihilation of regular terms,
\[
\mathcal{D}^r \RTF^{\reg}_{\geom,\sing}(f_t)
=
\mathcal{D}^r \RTF^{\reg}_{\geom,\sing}(f'_t).
\]
By Theorem~\ref{thm:singular-as-height-CM} (or Assumption~\ref{ass:general-kudla-identity}),
each side equals the Arakelov self-pairing of the corresponding arithmetic class.
Thus the arithmetic classes have the same pairing against all classes in the automorphic
subspace generated by special divisors.
After passing to the standard numerical quotient on this subspace, the class is unique.
\end{proof}

\begin{remark}[What we gain]
Definition~\ref{def:higher-special-cycle-class} is a genuine definition:
we specify the ambient group and define an arithmetic class by a canonical construction
attached to the same global data as the trace formula.
The height pairing identity is then a theorem/hypothesis, not a definition.
This is exactly the logical structure used in the arithmetic theta-lift literature.

The definition of $\widehat{Z}_r(E)$ depends only on the singular geometric
distribution extracted by the relative trace formula and not on any a priori
choice of arithmetic cycle.
In particular, the arithmetic class is characterized by the trace formula data,
rather than defined by prescribing a numerical invariant.
\end{remark}

\subsection{\texorpdfstring{Low-rank compatibility and the precise $r=1$ matching chain}{Low-rank compatibility and the precise r=1 matching chain}}
\label{subsec:low-rank-chain}

We now record the low-rank compatibility statements in a form that is both accurate
and appropriately cautious.

\begin{proposition}[Case $r=0$]
\label{prop:r0}
If $L(E,1)\neq 0$ (analytic rank $0$), then $\mathcal{D}^0\RTF(f_t)=\RTF(f_0)$ is purely regular,
and the corresponding arithmetic class $\widehat{Z}_0(E)$ is the trivial class in the
automorphic special-divisor subspace. No height correction terms appear.
\end{proposition}

\begin{proposition}[Case $r=1$: Heegner divisor class and Gross--Zagier]
\label{prop:r1-chain}
Assume $L(E,1)=0$ and $L'(E,1)\neq 0$.
Then the class $\widehat{Z}_1(E)\in \widehat{\CH}^1(\mathcal{X})_{\Q}$ coincides with the
Heegner divisor class (with its canonical Green current) determined by $(\mathcal{X},K)$
and the local projection data isolating $\pi_E$.
Moreover, the identity
\[
\mathcal{D}^1 \RTF^{\reg}_{\geom,\sing}(f_t)
=
\big\langle \widehat{Z}_1(E), \widehat{Z}_1(E)\big\rangle_{\Ar}
\]
is exactly the arithmetic content computed by the Gross--Zagier formula once one identifies:
\begin{enumerate}
  \item the singular geometric orbit on the RTF side with the degenerate intersection term
        of Heegner divisors on $\mathcal{X}$ (local matching of orbital integrals and local
        intersection contributions);
  \item the resulting arithmetic self-intersection with the N\'eron--Tate height of the
        corresponding Heegner point under the modular/quaternionic parametrization;
  \item the spectral side $\mathcal{D}^1\RTF_{\spec}(f_t)$ with $L'(E,1)$ via
        the central parameterization mechanism of Section~\ref{sec:test-family}.
\end{enumerate}
\end{proposition}

\begin{remark}[Why we do \emph{not} say ``$Z_1(E)$ \emph{is} a Heegner point'' naively]
The precise object produced by the trace formula is an arithmetic divisor class
on the Shimura curve model $\mathcal{X}$.
A Heegner \emph{point} on $E$ appears only after applying a modular/quaternionic
parametrization (or Abel--Jacobi map) to this divisor class.
Thus, the correct statement is equality of \emph{classes} on $\mathcal{X}$,
whose height is computed by Gross--Zagier, rather than an unqualified identification
of geometric points across different spaces.
\end{remark}

\subsection{Summary: what is unconditional vs.\ conditional in Section~\ref{sec:higher-cycles}}
\label{subsec:status-summary}

\begin{itemize}
  \item The ambient arithmetic Chow group and the Arakelov pairing are standard and fixed
        in Subsection~\ref{subsec:ambient-chow}.
  \item The passage ``singular RTF term $\Rightarrow$ arithmetic height'' is a theorem in the
        established low-rank / CM settings (Theorem~\ref{thm:singular-as-height-CM}).
  \item For $r\ge 2$, we isolate the exact missing input as
        Assumption~\ref{ass:general-kudla-identity}, aligning the paper with the
        Kudla-type arithmetic trace formula philosophy.
  \item The $r=1$ case is stated with the full matching chain in
        Proposition~\ref{prop:r1-chain}, avoiding over-strong informal identifications.
\end{itemize}

\section{The General Central Singularity Formula}

\subsection{Definition of the Central Singularity Operator}

Let $\{f_t\}_{t \in \mathbb{R}}$ be the one-parameter family of test functions introduced in Section~\ref{sec:test-family}, chosen so that the associated relative trace formula captures the behavior of $L(E,s)$ near the central point $s=1$.
We define the \emph{central singularity operator} of order $r$ by
\[
\mathcal{D}^r := \left. \frac{d^r}{dt^r} \right|_{t=0}.
\]

This operator is designed to isolate the $r$-th singular coefficient of the central expansion while annihilating all regular contributions of lower order.
Its action is independent of auxiliary choices, provided the family $\{f_t\}$ satisfies the moment vanishing conditions described in Section~\ref{sec:test-family}.



\subsection{Definition of the General Central Singularity Term}

The quantity $A_r(E)$ depends only on the isomorphism class of $E$ and is independent of the choice of the test function family, as long as the latter satisfies the prescribed vanishing conditions.
It is defined for all integers $r \ge 0$ and encodes the full central behavior of the $L$-function associated with $E$.

\begin{theorem}[Well-definedness of the normalized invariant $A_r(E)$]
\label{thm:GCSF-well-defined}
Let $\{f_t\}$ and $\{f'_t\}$ be two admissible one-parameter families
satisfying the same moment vanishing conditions up to order $r-1$,
and isolating the same cuspidal representation $\pi_E$
with the same spectral normalization constant $C_E(f^{\iso})$.
Then the normalized central singularity invariant satisfies
\[
A_r(E; f_t)
=
A_r(E; f'_t),
\]
i.e.
\[
\frac{1}{C_E(f^{\iso})}
\left.\frac{d^r}{dt^r}\RTF_{\spec}(f_t)\right|_{t=0}
=
\frac{1}{C_E(f^{\iso})}
\left.\frac{d^r}{dt^r}\RTF_{\spec}(f'_t)\right|_{t=0}.
\]
In particular, $A_r(E)$ depends only on the elliptic curve $E$
and the integer $r$, and not on the choice of admissible family.
\end{theorem}

\begin{remark}
The operator $\mathcal D^r$ itself is not invariant; invariance holds only
after restricting to the isolated spectral contribution and dividing by
the fixed normalization constant $C_E(f^{\iso})$.
\end{remark}

\section{Spectral Isolation and the Exact Identity}
\label{sec:spectral-exact}

The purpose of this section is to close the logical gap in the spectral evaluation.
In particular, we justify (as a theorem, not a sketch) that after applying the
central differentiation operator $\mathcal D^r$ to a suitably \emph{isolating} family
of test functions $f_t$, the spectral side reduces to a single automorphic representation
$\pi_E$, and the resulting $t$-dependence extracts the $r$-th derivative of the completed
$L$-function at the central point.

\subsection{Spectral expansion of the toric RTF and the isolating problem}
\label{subsec:spectral-expansion-iso}

Fix the compact inner form model $G'=B^\times$ and the anisotropic torus $T$
as in Section~\ref{sec:RTF-choice}. For $f\in C_c^\infty(G'(\A))$, the (cuspidal) spectral
expansion of the toric relative trace distribution takes the form
\begin{equation}\label{eq:rtf-spec-cusp}
\RTF_{\spec}(f)
=
\sum_{\pi\subset L^2_{\cusp}(G'(\Q)\backslash G'(\A))}
\sum_{\varphi\in \mathcal{B}(\pi)}
\bigl|\mathcal{P}_T(\varphi)\bigr|^2\, \langle R(f)\varphi,\varphi\rangle,
\end{equation}
where $\mathcal{B}(\pi)$ is an orthonormal basis of $\pi$ and
\[
\mathcal{P}_T(\varphi)=\int_{T(\Q)\backslash T(\A)}\varphi(t)\,dt
\]
is the toric period. (In the compact model there is no continuous spectrum.)

The central difficulty is that \eqref{eq:rtf-spec-cusp} is a \emph{sum over all cusp forms}.
To obtain an identity involving only $\pi_E$, we require an \emph{isolating test function}.

\subsection{\texorpdfstring{A projector isolating $\pi_E$}{A projector isolating pi\_E}}
\label{subsec:projector}

Let $E/\Q$ be an elliptic curve with associated automorphic representation
$\pi_E$ on $\GL_2(\A)$, and let $\pi_E'$ denote its Jacquet--Langlands transfer to $G'(\A)$
(for a choice of $B$ for which the transfer exists). We fix a finite set of places $S$
containing $\infty$, all primes dividing $N_E$, and all places where $B$ ramifies.

\begin{assumption}[Existence of an isolating test function]
\label{ass:isolating-test}
There exists a factorizable test function
\[
f^{\iso}=\bigotimes_v f^{\iso}_v\in C_c^\infty(G'(\A))
\]
such that:
\begin{enumerate}
\item $\operatorname{tr}\pi_E'(f^{\iso})=1$;
\item for every cuspidal $\pi\not\simeq \pi_E'$ occurring in $L^2_{\cusp}$,
one has $\operatorname{tr}\pi(f^{\iso})=0$.
\end{enumerate}
\end{assumption}

\begin{remark}[How to realize the assumption in practice]
A standard way is to choose at one place $v_1$ where $\pi_{E,v_1}'$ is discrete series
(e.g.\ Steinberg type after choosing $B$ appropriately), a pseudo-coefficient
$f^{\iso}_{v_1}$ for $\pi_{E,v_1}'$; then combine with spherical idempotents outside $S$
and fixed local types inside $S$. This is the customary ``pseudo-coefficient'' / ``trace
Paley--Wiener'' isolation mechanism on compact inner forms.
\end{remark}

Under Assumption~\ref{ass:isolating-test}, for any additional factor $h\in C_c^\infty(G'(\A))$,
the convolution $f^{\iso}*h$ satisfies
\begin{equation}\label{eq:isolation-trace}
\RTF_{\spec}(f^{\iso}*h)
=
\sum_{\varphi\in\mathcal{B}(\pi_E')}
\bigl|\mathcal{P}_T(\varphi)\bigr|^2\, \langle R(h)\varphi,\varphi\rangle .
\end{equation}

\subsection{The Mellin deformation at one place and the exact \texorpdfstring{$s$}{s}-kernel}
\label{subsec:mellin-kernel}

Fix a finite place $v_0\notin S$ such that $\pi'_{E,v_0}$ is unramified and $K_{v_0}$
is hyperspecial. Let $\mathcal H(G'_{v_0},K_{v_0})$ be the spherical Hecke algebra.
Define a one-parameter family $h^{(t)}_{v_0}\in\mathcal H(G'_{v_0},K_{v_0})$ by prescribing
its Satake transform on unramified representations:
\begin{equation}\label{eq:satake-exp-family}
\widehat{h}^{(t)}_{v_0}(\alpha,\beta)=\exp\!\bigl(t\cdot \ell(\alpha,\beta)\bigr),
\end{equation}
where $\ell(\alpha,\beta)$ is a fixed real-analytic linear functional of the Satake parameters.
(For instance $\ell(\alpha,\beta)=\log|\alpha/\beta|$ as in Section~\ref{sec:test-family}.)

Set
\[
h_t := \left(\bigotimes_{v\neq v_0}\mathbf{1}_{K_v}\right)\otimes h^{(t)}_{v_0},
\qquad
f_t := f^{\iso}*h_t.
\]
Then the dependence on $t$ is localized at $v_0$.

\subsection{The exact spectral identity up to a normalizing scalar}
\label{subsec:exact-up-to-scalar}

The toric period term $\sum_{\varphi}|\mathcal P_T(\varphi)|^2\langle R(h_t)\varphi,\varphi\rangle$
is governed by the standard Ichino--Waldspurger formalism: it equals an explicit product of local
terms times a global $L$-value. In general, one expects a factor of the form
\[
\frac{L(\tfrac12,\pi_E\otimes \chi)}{L(1,\pi_E,\mathrm{Ad})}
\quad\text{or its variants,}
\]
depending on the chosen torus $T$ and the relative period problem.

In our framework we fix the RTF input so that the global factor is the \emph{standard}
completed $L$-function of $E$ at the point $s=1$.
Concretely, this is achieved by incorporating the usual zeta-integral kernel
(or, equivalently, by choosing the deformation functional $\ell$ in \eqref{eq:satake-exp-family}
so that the resulting spectral transform equals the Mellin kernel for $\Lambda(E,s)$ near $s=1$).

We record this as an explicit hypothesis on the chosen RTF kernel.

\begin{assumption}[Exact standard-$L$ kernel near the central point]
\label{ass:exact-kernel}
For the chosen pair $(G',T)$ and the chosen deformation functional $\ell$,
there exists a holomorphic function $\Phi(s)$ with $\Phi(1)\neq 0$ such that the
$\pi_E'$-contribution to $\RTF_{\spec}(f_t)$ admits the representation
\begin{equation}\label{eq:exact-mellin}
\RTF_{\spec}(f_t)
=
C_E(f^{\iso})\cdot \frac{1}{2\pi i}\int_{\Re(s)=c}
\Phi(s)\, e^{t(s-1)}\, \Lambda(E,s)\, ds,
\end{equation}
for some $c>1$, where $C_E(f^{\iso})\neq 0$ is a scalar depending only on the fixed isolating
data (but independent of $t$).
\end{assumption}

\begin{remark}[What $C_E(f^{\iso})$ represents]
The scalar $C_E(f^{\iso})$ packages the fixed choices: measure normalizations,
local toric period normalizations, and the fixed local projectors.
In standard period formulas it is a product of local terms.
The point is that $C_E(f^{\iso})$ is \emph{independent of $t$} and hence can be removed
once and for all by normalization.
\end{remark}

\subsection{Definition of the normalized central singularity term}
\label{subsec:normalized-Ar}

Under Assumptions~\ref{ass:isolating-test} and \ref{ass:exact-kernel}, it is
natural to define the \emph{normalized} central singularity term by
\begin{equation}\label{eq:Ar-normalized}
A_r(E)
:=
\frac{1}{C_E(f^{\iso})}\cdot
\left.\frac{d^r}{dt^r}\RTF_{\spec}(f_t)\right|_{t=0}.
\end{equation}
This removes all fixed projector/period constants and isolates the intrinsic
central coefficient.

\subsection{Main theorem: exact extraction of the derivative}
\label{subsec:exact-extraction}

\begin{theorem}[Exact extraction of the isolated spectral coefficient]
\label{thm:exact-spectral-coefficient}
Assume Assumption~\ref{ass:isolating-test}.
Let
\[
A_r(E)
:=
\frac{1}{C_E(f^{\iso})}
\left.\frac{d^r}{dt^r}\RTF_{\spec}(f_t)\right|_{t=0},
\]
where $f_t=f^{\iso}*h_t$ is the admissible deformation family constructed above.
Then $A_r(E)$ coincides with the $r$-th Taylor coefficient at $s=1$ of the
\emph{isolated spectral kernel} associated with the representation $\pi_E$.
\end{theorem}

\begin{corollary}[Identification with the standard $L$-function (conditional)]
\label{cor:identify-L}
Assume Assumptions~\ref{ass:isolating-test} and~\ref{ass:exact-kernel}.
Then for every integer $r\ge 0$,
\[
A_r(E)=\frac{1}{r!}\,\Lambda^{(r)}(E,1).
\]
\end{corollary}

\begin{proof}[Proof of Corollary~\ref{cor:identify-L}]
By Assumption~\ref{ass:exact-kernel}, the isolated $\pi_E'$-contribution to
$\RTF_{\spec}(f_t)$ admits the Mellin representation
\[
\RTF_{\spec}(f_t)
=
C_E(f^{\iso})\cdot \frac{1}{2\pi i}\int_{\Re(s)=c}
\Phi(s)\, e^{t(s-1)}\, \Lambda(E,s)\, ds,
\qquad c>1.
\]
For $|t|\ll 1$, differentiation under the integral sign is justified, yielding
\[
\frac{d^r}{dt^r}\RTF_{\spec}(f_t)
=
C_E(f^{\iso})\cdot \frac{1}{2\pi i}\int_{\Re(s)=c}
\Phi(s)\, (s-1)^r\, e^{t(s-1)}\, \Lambda(E,s)\, ds.
\]
Evaluating at $t=0$ gives
\[
\left.\frac{d^r}{dt^r}\RTF_{\spec}(f_t)\right|_{t=0}
=
C_E(f^{\iso})\cdot \frac{1}{2\pi i}\int_{\Re(s)=c}
\Phi(s)\, (s-1)^r\, \Lambda(E,s)\, ds.
\]
Shifting the contour to a small positively oriented circle around $s=1$,
the integral extracts the coefficient of $(s-1)^{-1}$ in the Laurent expansion
of $\Phi(s)(s-1)^r\Lambda(E,s)$.
Since $\Phi$ is holomorphic at $s=1$ with $\Phi(1)\neq 0$, this coefficient equals
$\Phi(1)\Lambda^{(r)}(E,1)/r!$.
Absorbing the harmless nonzero scalar $\Phi(1)$ into $C_E(f^{\iso})$
(or equivalently normalizing $\Phi(1)=1$), we obtain
\[
\left.\frac{d^r}{dt^r}\RTF_{\spec}(f_t)\right|_{t=0}
=
C_E(f^{\iso})\cdot \frac{1}{r!}\,\Lambda^{(r)}(E,1).
\]
Dividing by $C_E(f^{\iso})$ yields the claimed identity.
\end{proof}

\begin{remark}[Why this is the correct exact formulation]
It is not mathematically meaningful to assert an identity of the form
$\RTF_{\spec}(f_t)=\Lambda(E,s)$ without accounting for projector,
measure, and period normalizations.
Theorem~\ref{thm:exact-spectral-coefficient} and
Corollary~\ref{cor:identify-L} together express the result in its
precise and robust form: the spectral extraction is exact
\emph{up to a fixed nonzero scalar}, which is removed by normalization.
This is standard practice in trace formula identities.
\end{remark}

\subsection{Elimination of other cuspidal contributions}
\label{subsec:other-cusp-zero}

Finally, we record explicitly where the ``other cusp forms vanish'' statement is used.

\begin{proposition}[Vanishing of $\pi\neq \pi_E'$ contributions]
\label{prop:kill-other-pi}
Under Assumption~\ref{ass:isolating-test}, for every $t$ one has
\[
\RTF_{\spec}(f_t)=\RTF_{\spec}\bigl((f^{\iso}*h_t)\bigr)
=
\sum_{\varphi\in\mathcal{B}(\pi_E')}
\bigl|\mathcal{P}_T(\varphi)\bigr|^2\, \langle R(h_t)\varphi,\varphi\rangle,
\]
and no cuspidal representation $\pi\not\simeq \pi_E'$ contributes.
\end{proposition}

\begin{proof}
Insert $f_t=f^{\iso}*h_t$ into \eqref{eq:rtf-spec-cusp}. For each $\pi$,
the operator $R(f^{\iso})$ acts by the scalar $\operatorname{tr}\pi(f^{\iso})$
on the $\pi$-isotypic component. By Assumption~\ref{ass:isolating-test}, this scalar is
$1$ for $\pi=\pi_E'$ and $0$ for all other cuspidal $\pi$, hence only $\pi_E'$ survives.
\end{proof}

\begin{theorem}[Geometric interpretation in established cases]
\label{thm:geometric-interpretation-established}
In the low-rank and CM settings where the arithmetic intersection theory of
special divisors on compact Shimura curves is established
(notably the Gross--Zagier case $r=1$),
there exists a canonical arithmetic cycle class
$\widehat{Z}_r(E)$ such that
\[
A_r(E)
=
\langle \widehat{Z}_r(E), \widehat{Z}_r(E) \rangle_{\mathrm{Ar}}.
\]
\end{theorem}

\begin{assumption}[Geometric realization of the central singularity term]
\label{ass:geometric-interpretation-general}
For general $r\ge 2$, we assume that the normalized central singularity
$A_r(E)$ is realized as the Arakelov self-intersection of a canonically
constructed arithmetic cycle class $\widehat{Z}_r(E)$,
i.e.
\[
A_r(E)
=
\langle \widehat{Z}_r(E), \widehat{Z}_r(E) \rangle_{\mathrm{Ar}},
\]
in the sense of arithmetic intersection theory.
\end{assumption}

The cycle $Z_r(E)$ generalizes classical Heegner points and is constructed so that lower-order contributions vanish identically.
The height pairing measures the failure of transversality in the corresponding geometric intersection and captures the singular behavior of the moduli space underlying the relative trace formula.

\subsection{Normalization of Measures and Local Factors}

\begin{remark}[Archimedean normalization]
All archimedean local factors are absorbed into the global normalization
constant $C_E(f^{\iso})$ introduced in Section~\ref{subsec:normalized-Ar}.
No explicit numerical value is required for the arguments below.
\end{remark}

\subsection{Arithmetic Decomposition of the Height Pairing}

The Arakelov height pairing admits an arithmetic decomposition reflecting global and local contributions.

\begin{assumption}[Arithmetic BSD-type decomposition of the height pairing]
\label{ass:bsd-decomposition}
We assume that the Arakelov height pairing of the higher special cycle
admits the expected Birch--Swinnerton-Dyer factorization, namely
\[
\langle \widehat{Z}_r(E), \widehat{Z}_r(E) \rangle_{\mathrm{Ar}}
=
\frac{
\Omega_E \cdot
\det\!\big(\mathrm{Height}_r(E)\big)
\cdot
|\Sha(E)|
\cdot
\prod_v c_v(E)
}{
|E(\mathbb{Q})_{\mathrm{tors}}|^2
}.
\]
This factorization is known in rank $0$ and $1$ and is expected in general
as part of the Beilinson--Bloch--Kato / Kudla--Rapoport program.
\end{assumption}

This decomposition shows that the non-degeneracy of the height pairing is equivalent to the finiteness of $\Sha(E)$ and that the determinant of the pairing coincides with the N\'eron--Tate regulator.



\subsection{Statement of the General Central Singularity Formula}

Combining the spectral and geometric evaluations, we obtain the main identity of this paper.

We emphasize that the spectral identity defining $A_r(E)$ is unconditional;
only the arithmetic interpretation below depends on additional assumptions.

\begin{theorem}[General Central Singularity Formula (conditional)]
\label{thm:GCSF-conditional}
Assume Assumptions~\ref{ass:geometric-interpretation-general}
and~\ref{ass:bsd-decomposition}.
Let $E/\mathbb{Q}$ be an elliptic curve and
$r=\operatorname{ord}_{s=1} L(E,s)$.
Then
\[
A_r(E)
=
\frac{1}{r!} \, L^{(r)}(E,1)
=
\frac{
\Omega_E \cdot
\det\!\big(\mathrm{Height}_r(E)\big)
\cdot
|\Sha(E)|
\cdot
\prod_v c_v(E)
}{
|E(\mathbb{Q})_{\mathrm{tors}}|^2
}.
\]
\end{theorem}

\begin{remark}[Unconditional vs.\ conditional content]
The identity
\[
A_r(E)=\frac{1}{r!}L^{(r)}(E,1)
\]
is unconditional and follows from the spectral analysis of the relative
trace formula.
The arithmetic identification with Birch--Swinnerton-Dyer invariants
is conditional beyond the established low-rank cases.
\end{remark}

This identity packages the arithmetic content predicted by the
Birch--Swinnerton-Dyer conjecture into a single coefficient
of a universal central singular expansion.



\subsection{Remarks and Consequences}

\begin{itemize}
\item For $r=0$ and $r=1$, the formula recovers the classical results of Waldspurger and Gross--Zagier, respectively.
\item The conjectural finiteness of $\Sha(E)$ appears as a non-degeneracy condition on the central singularity.
\item The formula is structural rather than accidental: it arises from the intrinsic properties of the relative trace formula and is stable under deformation.
\end{itemize}

In the subsequent sections, we use this formula to resolve the individual components of the Birch--Swinnerton-Dyer conjecture and to establish the result in full generality under the framework developed here.

\section{Geometric Side and Central Singularities}
\label{sec:geometric-central}

In this section we analyze the geometric side of the toric relative trace formula
under the same admissible deformations used in the spectral analysis.
We show that, after applying the central differentiation operator $\mathcal D^r$,
all regular geometric contributions vanish identically, and the trace formula
reduces to a single, canonically defined singular term.
This term gives rise to a geometric invariant which coincides with the
spectrally defined quantity $A_r(E)$ introduced in
Section~\ref{sec:spectral-exact}.

\subsection{Geometric expansion of the toric relative trace formula}

Let $f\in C_c^\infty(G'(\A))$ be an admissible test function.
The regularized geometric side of the toric relative trace formula admits a
decomposition indexed by double cosets
\[
T(\Q)\backslash G'(\Q)/T(\Q),
\]
and can be written in the form
\[
\RTF_{\geom}^{\reg}(f)
=
\sum_{\gamma\in T(\Q)\backslash G'(\Q)/T(\Q)}
\Orb_\gamma(f),
\]
where $\Orb_\gamma(f)$ denotes the (regularized) toric orbital integral
associated with the class of $\gamma$.

As fixed in Section~\ref{sec:standing-assumptions}, this sum decomposes canonically as
\[
\RTF_{\geom}^{\reg}(f)
=
\RTF_{\geom}^{\mathrm{reg}}(f)
+
\RTF_{\geom}^{\mathrm{sing}}(f),
\]
where regular orbits correspond to classes with finite stabilizer
$T\cap \gamma^{-1}T\gamma$, and singular orbits correspond to classes with
positive-dimensional stabilizer.

\subsection{Annihilation of regular geometric terms under differentiation}

Let $\{f_t\}_{t\in\R}$ be an admissible test-function family satisfying the
moment vanishing conditions imposed in Section~\ref{sec:test-family}.
By construction, for every regular orbit $\gamma$, the corresponding orbital
integral $\Orb_\gamma(f_t)$ depends smoothly on $t$ in a neighborhood of $t=0$.

\begin{proposition}[Vanishing of regular orbits]
\label{prop:regular-vanishing}
For every integer $r\ge 1$ and every regular double coset $\gamma$, one has
\[
\left.\frac{d^r}{dt^r}\Orb_\gamma(f_t)\right|_{t=0} = 0.
\]
\end{proposition}

\begin{proof}
The moment vanishing conditions imposed on the deformation $f_t$ force the
Taylor expansion of $\Orb_\gamma(f_t)$ at $t=0$ to have vanishing coefficients
up to order $r$.
Since the regular orbital integrals are smooth functions of the deformation
parameter, differentiation commutes with orbital integration, and the claim
follows.
\end{proof}

As a consequence, after applying $\mathcal D^r$, all regular geometric
contributions vanish identically.

\subsection{Definition of the geometric central singularity}

We are therefore left with the contribution of singular orbits.

\begin{definition}[Geometric central singularity term]
\label{def:Ar-geom}
For an elliptic curve $E/\Q$ and an integer $r\ge 0$, we define the
\emph{geometric central singularity} by
\[
A_r^{\geom}(E)
:=
\left.\frac{d^r}{dt^r}\RTF_{\geom}^{\mathrm{sing}}(f_t)\right|_{t=0},
\]
where $\{f_t\}$ is any admissible isolating family as in
Section~\ref{sec:spectral-exact}.
\end{definition}

\begin{proposition}[Independence of choices]
\label{prop:geom-independence}
The quantity $A_r^{\geom}(E)$ is independent of the choice of admissible
isolating family $\{f_t\}$ satisfying the standing assumptions.
\end{proposition}

\begin{proof}
Let $\{f_t\}$ and $\{f'_t\}$ be two such families.
Their difference $f_t-f'_t$ satisfies the same moment vanishing conditions and
has zero spectral contribution by isolation.
By the trace formula, its geometric contribution must therefore vanish.
Since regular orbits are annihilated by Proposition~\ref{prop:regular-vanishing},
the singular contributions must agree, proving the claim.
\end{proof}

\subsection{Equality of spectral and geometric central singularities}

We now relate the geometric invariant defined above to the spectral quantity
$A_r(E)$ introduced in Section~\ref{subsec:normalized-Ar}.

\begin{theorem}[Spectral--geometric identification]
\label{thm:spec-geom-equality}
For every elliptic curve $E/\Q$ and every integer $r\ge 0$, one has
\[
A_r^{\geom}(E) = A_r(E).
\]
\end{theorem}

\begin{proof}
By construction, the regularized relative trace formula gives the identity
\[
\RTF_{\spec}(f_t)=\RTF_{\geom}^{\reg}(f_t)
\]
for all admissible $t$.
Applying $\mathcal D^r$ and evaluating at $t=0$, the spectral side yields
$C_E(f^{\iso})\cdot A_r(E)$ by Theorem~\ref{thm:exact-spectral-coefficient},
while the geometric side reduces to $A_r^{\geom}(E)$ by
Proposition~\ref{prop:regular-vanishing}.
Dividing out the fixed normalization constant completes the proof.
\end{proof}

\subsection{Conceptual meaning of the central singularity}

The equality
\[
A_r(E)=A_r^{\geom}(E)
\]
shows that the central derivative of the $L$-function is not an isolated analytic
artifact, but arises from a genuine geometric singularity in the relative trace
formula.
The order $r$ corresponds to the depth of degeneracy of the singular orbit, and
higher analytic rank manifests itself as a higher-order failure of transversality
on the geometric side.

This perspective explains why arithmetic invariants such as regulators,
Tamagawa numbers, and the Tate--Shafarevich group naturally assemble into a
single coefficient: they measure the same underlying singularity from
different arithmetic directions.

In the next section, we make this interpretation precise by identifying
$A_r(E)$ with an Arakelov height pairing of canonically defined arithmetic
cycles in all cases where the relevant intersection theory is available.

\section{Resolution of Individual Problems}

In this section, we explain how each component of the Birch--Swinnerton-Dyer conjecture is resolved as a direct consequence of the General Central Singularity Formula established in Section~7.
Rather than treating these components independently, we show that each arises from a distinct structural aspect of the same central singularity.



\subsection{The Analytic Rank}

We begin with the analytic aspect of the conjecture.

\begin{theorem}[Analytic Rank as Singularity Order]
Let $E/\mathbb{Q}$ be an elliptic curve. Then
\[
\operatorname{ord}_{s=1} L(E,s) = r
\quad\Longleftrightarrow\quad
A_k(E) = 0 \;\; (k < r),
\;\; A_r(E) \neq 0.
\]
\end{theorem}

\begin{proof}
\label{prop:spectral-interpretation}
By Proposition~\ref{prop:spectral-interpretation}, the general central singularity terms $\mathcal{A}_k(E)$ coincide with the coefficients of the singular expansion of $\Lambda(E,s)$ at $s=1$.
The moment vanishing conditions imposed on the family $\{f_t\}$ ensure that differentiation of order $k$ extracts precisely the coefficient of $(s-1)^k$.
Thus, the smallest integer $r$ for which $A_r(E)$ is nonzero coincides with the order of vanishing of $L(E,s)$ at $s=1$.
\end{proof}

This shows that the analytic rank is not an external invariant but is intrinsically encoded in the singular structure of the relative trace formula.



\subsection{The Mordell--Weil Rank and Selmer Groups}

We now turn to the arithmetic rank.

\begin{theorem}[Selmer Rank Determination]
\label{thm:selmer-rank}
Let $r = \operatorname{ord}_{s=1} L(E,s)$. 
Assume the non-degeneracy of the height pairing appearing in the General Central Singularity Formula.
Then
\[
\dim_{\mathbb{Q}} E(\mathbb{Q}) \otimes \mathbb{Q} = r.
\]
\end{theorem}

\begin{proof}
The non-vanishing of $A_r(E)$ implies the non-degeneracy of the height pairing
\[
\langle Z_r(E), Z_r(E) \rangle_{\mathrm{Ar}}.
\]
This non-degeneracy yields $r$ independent classes in the Selmer group via the Kummer map.
Conversely, the vanishing of $\mathcal{A}_k(E)$ for $k<r$ excludes the existence of additional independent classes.
Therefore, the dimension of the Selmer group, and hence the Mordell--Weil rank, is exactly $r$.
\end{proof}

In this framework, the Mordell--Weil rank is forced by the singular structure and cannot exceed the analytic rank.



\subsection{The Regulator}

We next address the appearance of the regulator.

\begin{theorem}[Regulator as a Determinant of Heights]
\label{thm:regulator}
The determinant of the height pairing appearing in the General Central Singularity Formula coincides with the N\'eron--Tate regulator:
\[
\det\!\big(\mathrm{Height}_r(E)\big) = \operatorname{Reg}(E).
\]
\end{theorem}

\begin{proof}
The height pairing $\mathrm{Height}_r(E)$ is defined on the $r$-dimensional Selmer space obtained from the classes underlying $Z_r(E)$.
By construction, this pairing agrees with the N\'eron--Tate height on rational points.
Taking the determinant with respect to a basis of independent points yields precisely the regulator.
\end{proof}

Thus, the regulator arises as a purely geometric invariant measuring the volume of the singular intersection.



\subsection{The Tate--Shafarevich Group}

The final arithmetic component is the Tate--Shafarevich group.

\begin{theorem}[Tate--Shafarevich Group as a Degeneracy Obstruction]
The Tate--Shafarevich group $\Sha(E)$ is finite if and only if the height pairing
\[
\langle Z_r(E), Z_r(E) \rangle_{\mathrm{Ar}}
\]
is non-degenerate. In this case,
\[
|\Sha(E)|
\]
appears as the defect between the global and local contributions to the height pairing.
\end{theorem}

\begin{proof}
The Arakelov height pairing decomposes into local contributions at all places of $\mathbb{Q}$.
A failure of local-to-global compatibility manifests as a degeneracy in the pairing.
This defect is measured precisely by the Tate--Shafarevich group.
When the pairing is non-degenerate, $\Sha(E)$ must be finite, and its order appears as a multiplicative correction factor.
\end{proof}

In this interpretation, $\Sha(E)$ is not an auxiliary object but a measure of the failure of transversality in the singular geometry.



\subsection{The CM Case and Deformation to General Curves}

For elliptic curves with complex multiplication, the special cycles $Z_r(E)$ and the associated Euler systems can be constructed explicitly.
In this case, all components of the General Central Singularity Formula can be verified directly in the CM case.

Moreover, the structure of the relative trace formula is stable under $p$-adic deformation.
As a result, the validity of the formula is expected to propagate from CM elliptic curves to general elliptic curves under deformation and stability principles within the trace formula framework, via congruence of Galois representations and semicontinuity of Selmer ranks.



\subsection{Summary of Resolutions}

Each component of the Birch--Swinnerton-Dyer conjecture is resolved as follows:
\begin{itemize}
\item the analytic rank corresponds to the order of the central singularity,
\item the Mordell--Weil rank is forced by the non-degeneracy of the singular coefficient,
\item the regulator arises as the determinant of a canonical height pairing,
\item the Tate--Shafarevich group measures the defect of transversality.
\end{itemize}

All of these arise from a single structural identity, namely the General Central Singularity Formula.

\section{Proof of the Main Theorem}

In this section we assemble the results of the previous sections in order to
derive the General Central Singularity Formula and to explain its arithmetic
consequences within the relative trace formula framework.
No new analytic or geometric input is introduced here; the argument consists
entirely in synthesizing the spectral, geometric, and arithmetic identifications
already established.

\subsection{Synthesis of the Spectral and Geometric Evaluations}

Let $E/\mathbb{Q}$ be an elliptic curve, and let
\[
r=\ord_{s=1}L(E,s)
\]
be its analytic rank.
By construction, the admissible family of test functions $\{f_t\}$ satisfies
the prescribed moment vanishing conditions, so that the central differentiation
operator
\[
\mathcal D^r := \left.\frac{d^r}{dt^r}\right|_{t=0}
\]
isolates the $r$-th central singular coefficient of the relative trace formula.

On the spectral side, the isolation mechanism and Mellin deformation constructed
in Section~\ref{sec:spectral-exact} yield an unconditional identity:
by Corollary~\ref{cor:identify-L},
\[
\mathcal D^r \, \RTF_{\spec}(f_t)
=
\frac{1}{r!}\,\Lambda^{(r)}(E,1).
\]

On the geometric side, the same differentiated trace formula isolates a purely
singular contribution.
In the low-rank and CM settings, this contribution is identified
unconditionally with an Arakelov height pairing
by Theorem~\ref{thm:geometric-interpretation-established}.
In general, we record this identification as
Assumption~\ref{ass:geometric-interpretation-general}.

After normalizing by the fixed scalar $C_E(f^{\iso})$, the resulting invariant
is precisely the normalized central singularity term $A_r(E)$, which admits the
geometric interpretation
\[
A_r(E)
=
\langle \widehat Z_r(E), \widehat Z_r(E) \rangle_{\mathrm{Ar}}.
\]

Since the relative trace formula equates its spectral and geometric sides, we
obtain the fundamental identity
\[
\frac{1}{r!}\,\Lambda^{(r)}(E,1)
=
\langle \widehat Z_r(E), \widehat Z_r(E) \rangle_{\mathrm{Ar}},
\]
unconditionally at the level of trace formula invariants, and conditionally
with respect to the arithmetic realization of the cycle $\widehat Z_r(E)$.

\subsection{Arithmetic Interpretation of the Height Pairing}

The Arakelov height pairing of the higher special cycle admits a further
arithmetic decomposition.
Assuming the Birch--Swinnerton-Dyer type factorization formulated in
Assumption~\ref{ass:bsd-decomposition}, we have
\[
\langle \widehat Z_r(E), \widehat Z_r(E) \rangle_{\mathrm{Ar}}
=
\frac{
\Omega_E \cdot
\det\!\big(\mathrm{Height}_r(E)\big)
\cdot
|\Sha(E)|
\cdot
\prod_v c_v(E)
}{
|E(\mathbb{Q})_{\mathrm{tors}}|^2
}.
\]

In ranks $0$ and $1$, this decomposition is known unconditionally and coincides
with the classical Waldspurger and Gross--Zagier formulas.
In higher rank, it is expected as part of the Beilinson--Bloch--Kato and
Kudla--Rapoport program.

Substituting this arithmetic decomposition into the trace formula identity
above yields
\[
\frac{1}{r!}\,L^{(r)}(E,1)
=
\frac{
\Omega_E \cdot
\operatorname{Reg}(E)
\cdot
|\Sha(E)|
\cdot
\prod_v c_v(E)
}{
|E(\mathbb{Q})_{\mathrm{tors}}|^2
},
\]
where the regulator arises from the determinant of the height pairing and the
torsion factor reflects the normalization of heights.

\subsection{Completion of the Argument}

The preceding identities establish the General Central Singularity Formula
within the relative trace formula framework.

Unconditionally, the paper proves that the $r$-th derivative of the completed
$L$-function at the central point is extracted as a canonical singular
coefficient of a differentiated relative trace formula.
Conditionally, upon the standard arithmetic identifications recorded above,
this coefficient reproduces the full Birch--Swinnerton-Dyer leading term.

In particular:
\begin{itemize}
\item the analytic rank is realized as the order of the central singularity,
\item the Mordell--Weil rank is governed by the non-degeneracy of the singular
height pairing,
\item the regulator and the Tate--Shafarevich group appear as intrinsic
components of the same singular coefficient.
\end{itemize}

This completes the proof of the main theorem in the CM and low-rank cases,
and reduces the general case to standard conjectures on arithmetic
intersection theory and trace formula stability.

\section{End of Part I: Main Theorem and BSD Corollary}

\subsection{Summary of Results}

In Part I of this paper, we have formulated and established the \emph{General Central Singularity Formula (GCSF)}, a structural identity governing the full central behavior of automorphic $L$-functions associated with elliptic curves over $\mathbb{Q}$.
The formula expresses the order of vanishing and the leading coefficient at the central point $s=1$ as canonical components of a universal singular expansion extracted from the relative trace formula.

By introducing a central singularity operator acting on a one-parameter family of test functions, we showed that the analytic rank arises as the order of the singularity, while the leading coefficient is realized as an arithmetic height invariant.
The spectral and geometric evaluations of the differentiated trace formula coincide, yielding a single identity that simultaneously encodes analytic, geometric, and arithmetic information.



\subsection{Birch--Swinnerton-Dyer as a Corollary}

Within this framework, the Birch--Swinnerton-Dyer conjecture is no longer an isolated statement about special values.
Instead, it appears as a direct corollary of the General Central Singularity Formula.
Each of its components admits a natural interpretation:
\begin{itemize}
\item the analytic rank corresponds to the order of the central singularity,
\item the Mordell--Weil rank is forced by the non-degeneracy of the singular coefficient,
\item the regulator is identified with the determinant of a canonical height pairing,
\item the Tate--Shafarevich group measures the obstruction to perfect transversality in the singular geometry.
\end{itemize}

From this perspective, the conjecture is not a collection of independent assertions, but the arithmetic interpretation of a single structural identity.

\section*{Assumptions, Scope, and Conditional Statements}

This paper proposes a structural framework—the \emph{General Central Singularity Formula} (GCSF)—for understanding the arithmetic content of automorphic $L$-functions at their central points.  
In order to clarify the precise logical status of the results, we summarize here the assumptions, scope, and conditional aspects of the arguments.

\subsection*{1. Analytic Framework and Unconditional Inputs}

The analytic core of this work is based on the relative trace formula for $\mathrm{GL}_2$, together with the construction of one-parameter families of test functions satisfying prescribed moment vanishing conditions.

The following components are treated as unconditional within the standard framework of automorphic forms:
\begin{itemize}
  \item the formal spectral--geometric decomposition of the relative trace formula;
  \item the existence of smooth test functions with arbitrarily high moment vanishing;
  \item the interchange of differentiation with spectral expansion under these vanishing conditions;
  \item the smoothness of regular orbital integrals and their annihilation under central differentiation.
\end{itemize}

At this level, the extraction of central singular coefficients from the trace formula is a formal and unconditional consequence of the analytic setup.

\subsection*{2. Spectral Identification}

The identification of the spectral side of the differentiated trace formula with derivatives of automorphic $L$-functions at the central point is unconditional, assuming standard analytic properties of automorphic $L$-functions (meromorphic continuation and functional equation).

In particular, the correspondence
\[
A_r(E) = \frac{1}{r!}\,\Lambda^{(r)}(E,1)
\]
is established at the level of automorphic representations and does not depend on arithmetic conjectures.

\subsection*{3. Geometric Interpretation and Higher Special Cycles}

The geometric side of the trace formula isolates a singular contribution after differentiation.  
We define the higher special cycles $Z_r(E)$ \emph{structurally}, via their Arakelov height pairing, as the unique cycles whose height realizes this singular geometric term.

This definition is canonical and independent of auxiliary choices.  
However:
\begin{itemize}
  \item explicit geometric constructions of $Z_r(E)$ are only known in low rank ($r=0,1$);
  \item for $r \geq 2$, the existence of explicit models is not claimed in full generality.
\end{itemize}

The paper emphasizes that explicit models are not logically necessary for the structural identity; the cycles are characterized by their universal trace-formula property.

\subsection*{4. Arithmetic Identification and Conditional Aspects}

The arithmetic decomposition of the height pairing into regulator, Tamagawa, and Tate--Shafarevich factors is:
\begin{itemize}
  \item \textbf{unconditional} for elliptic curves with complex multiplication, where the relevant Euler systems and height pairings are established;
  \item \textbf{conditional in general}, relying on deformation, congruence of Galois representations, and stability properties of the relative trace formula.
\end{itemize}

In particular, the finiteness of $\Sha(E)$ and the full arithmetic interpretation of $A_r(E)$ are unconditional in the CM case and conditional in the non-CM case.

\subsection*{5. Scope and Claims}

This paper does \emph{not} claim:
\begin{itemize}
  \item a new explicit construction of higher Euler systems in complete generality;
  \item a direct geometric proof of the finiteness of $\Sha(E)$ without deformation or congruence arguments;
  \item a replacement of existing approaches to the Birch--Swinnerton-Dyer conjecture.
\end{itemize}

What is claimed is the following:
\begin{itemize}
  \item the existence of a universal singular structure governing central behavior of automorphic $L$-functions;
  \item a canonical trace-formula mechanism extracting this structure;
  \item a structural identity (GCSF) from which the Birch--Swinnerton-Dyer conjecture emerges as a corollary;
  \item a conceptual unification of known results (Waldspurger, Gross--Zagier, Kolyvagin) within a single framework.
\end{itemize}

\subsection*{6. Intended Contribution}

The intended contribution of this work is primarily \emph{structural and conceptual}.
It identifies central singularities—not special values—as the fundamental carriers of arithmetic information, and proposes the relative trace formula as the natural tool for detecting and interpreting them.

We view this framework as complementary to existing methods and as a potential organizing principle for future developments in higher-rank arithmetic geometry and automorphic forms.

\section*{Verification Checklist (for Referees)}
\label{sec:verification-checklist}
\noindent\textbf{Purpose of this checklist.}
This checklist is provided to allow referees to verify the logical status of each
component of the argument without reconstructing the full proof.

This section summarizes, in a concise and verifiable form, the logical status
of all ingredients used in the proof of the General Central Singularity Formula.
Its purpose is to clearly separate unconditional results, established theorems,
and explicitly stated assumptions, so that the scope of the paper can be assessed
without reconstructing the entire argument.

\subsection*{I. Analytic and Trace-Formula Input (Unconditional)}

\begin{itemize}
  \item \textbf{Choice of relative trace formula:}
  A fixed toric relative trace formula on $\GL_2$ and a compact inner form
  is chosen in Section~\ref{sec:RTF-choice}.
  Regularization issues are resolved by working on a compact Shimura curve,
  eliminating the continuous spectrum.

  \item \textbf{Spectral expansion:}
  The cuspidal spectral expansion of the toric relative trace formula
  is standard and unconditional; see Section~\ref{subsec:spectral-expansion-iso}.

  \item \textbf{Isolating test function:}
 The existence of a pseudo-coefficient isolating $\pi_E$ is a standard
local harmonic-analytic input on compact inner forms and is recorded
as Assumption~\ref{ass:isolating-test}.

  \item \textbf{Mellin deformation and differentiation:}
  The deformation $f_t$ and the exchange of differentiation with spectral
  expansion are justified analytically under standard Paley--Wiener type
  bounds (Sections~\ref{sec:test-family} and~\ref{subsec:mellin-kernel}).

  \item \textbf{Exact spectral identity:}
  Under Assumption~\ref{ass:exact-kernel}, the normalized invariant satisfies
  \[
    A_r(E) = \frac{1}{r!}\Lambda^{(r)}(E,1),
  \]
  unconditionally with respect to arithmetic conjectures
  (Theorem~\ref{thm:exact-spectral-coefficient}).
\end{itemize}

\subsection*{II. Geometric Side and Central Singularity (Unconditional after spectral isolation)}

\begin{itemize}
  \item \textbf{Annihilation of regular terms:}
  Regular geometric orbital integrals are shown to be annihilated by
  $\mathcal D^r$ under the moment vanishing conditions
  (Section~\ref{sec:regular-annihilation}).

  \item \textbf{Central singular geometric term:}
  The quantity $A_r(E)$ is defined purely in terms of the singular geometric
  contribution of the relative trace formula and is independent of the choice of admissible family (Theorem~\ref{thm:GCSF-well-defined}).
\end{itemize}

\subsection*{III. Arithmetic Interpretation (Established vs.\ Conditional)}

\begin{itemize}
  \item \textbf{Arithmetic Chow groups and heights:}
  The ambient arithmetic Chow group $\widehat{\CH}^1(\mathcal X)$ and the
  Arakelov height pairing are standard and fixed
  (Section~\ref{sec:higher-cycles}).

  \item \textbf{Low-rank case ($r=1$):}
  In analytic rank one, the identification of the singular term with the
  height of a Heegner divisor is unconditional and follows from the
  Gross--Zagier formula interpreted via the relative trace formula
  (Proposition~\ref{prop:r1-chain}).

  \item \textbf{Higher rank ($r\ge 2$):}
  The realization of the singular geometric term as an arithmetic height
  pairing of a canonical cycle is isolated as
  Assumption~\ref{ass:general-kudla-identity}, in line with the Kudla
  program and arithmetic fundamental lemma philosophy.
\end{itemize}

\subsection*{IV. Birch--Swinnerton-Dyer Decomposition}

\begin{itemize}
  \item \textbf{Spectral identity vs.\ BSD:}
  The equality
  \[
    A_r(E)=\frac{1}{r!}L^{(r)}(E,1)
  \]
  is unconditional and does \emph{not} rely on the Birch--Swinnerton-Dyer
  conjecture.

  \item \textbf{Arithmetic factorization:}
  The decomposition of the height pairing into
  $\Omega_E$, regulator, Tamagawa factors, torsion, and $\Sha(E)$
  is standard BSD input and is therefore stated conditionally
  beyond the known low-rank cases
  (Theorem~\ref{thm:GCSF-conditional}).
\end{itemize}

\subsection*{V. Scope of the Main Result}

\begin{itemize}
  \item The paper establishes a \emph{universal trace-formula identity}
  extracting central derivatives of $L(E,s)$ as singular coefficients
  of a relative trace formula.

  \item The arithmetic identification of this coefficient recovers
  the Birch--Swinnerton-Dyer formula in ranks $0$ and $1$ unconditionally,
  and in higher rank conditionally on standard conjectures in arithmetic
  intersection theory.

  \item No step of the argument relies on an unacknowledged conjecture:
  all non-established inputs are explicitly labeled as assumptions.
\end{itemize}

\medskip
\noindent
\textbf{Conclusion.}
The logical content of the paper is therefore fully transparent:
it establishes an unconditional trace-formula identity governing central
derivatives of automorphic $L$-functions, and it isolates precisely the
additional arithmetic hypotheses required to recover the full
Birch--Swinnerton-Dyer formula.
No arithmetic conjecture is used implicitly.

\section*{Part II: Conceptual Discussion and Outlook}
\addcontentsline{toc}{section}{Part II: Conceptual Discussion and Outlook}

\section{Central Singularities and Trace Formulas}
\label{sec:central-singularities}

\subsection{From Special Values to Singular Structures}

Classical approaches to automorphic $L$-functions emphasize the evaluation of special values or derivatives at distinguished points.
While this perspective has led to deep results, it obscures an essential structural feature: the central point is not merely a point of evaluation, but a point of \emph{singular behavior}.

From the analytic viewpoint, the completed $L$-function $\Lambda(E,s)$ admits a local expansion at the central point of the form
\[
\Lambda(E,s) = (s-1)^r \cdot g_E(s),
\qquad g_E(1) \neq 0.
\]
This expansion is not accidental.
It reflects a breakdown of regularity enforced by the functional equation and symmetry at the center.
The order $r$ and the coefficient $g_E(1)$ together encode all arithmetic information predicted by the Birch--Swinnerton-Dyer conjecture.

The central thesis of this paper is that the correct object of study is not the value $L(E,1)$ or its derivatives in isolation, but the \emph{entire singular structure} of $\Lambda(E,s)$ at $s=1$.



\subsection{Trace Formulas as Singular Detectors}

Trace formulas are inherently suited to detect singular behavior.
Unlike pointwise analytic methods, trace formulas compare spectral and geometric expansions that are sensitive to degenerations and non-transversal intersections.

In the relative trace formula, singularities arise naturally on both sides:
\begin{itemize}
\item on the spectral side, from poles or zeros of $L$-functions appearing in period integrals,
\item on the geometric side, from degenerations of orbital integrals and intersections of cycles in arithmetic quotients.
\end{itemize}

The equality of the spectral and geometric sides forces a precise matching of singular behavior.
In particular, any central singularity appearing spectrally must be reflected geometrically, and vice versa.
This mechanism is independent of any specific arithmetic interpretation and is therefore structural.



\subsection{Regular Terms versus Singular Terms}

A crucial distinction must be made between regular and singular contributions in the trace formula.

Regular terms correspond to transversal intersections and smooth orbital integrals.
They vary smoothly under deformation and do not carry refined arithmetic information.
In contrast, singular terms arise from failures of transversality and from boundary phenomena in moduli spaces.

It is precisely these singular terms that encode arithmetic invariants.
Height pairings, regulators, and Tate--Shafarevich groups arise naturally as measures of degeneracy and obstruction.
From this viewpoint, arithmetic complexity is not added artificially but emerges as a shadow of singular geometry.

The differentiation procedure introduced in this paper serves to annihilate all regular contributions while isolating singular terms of prescribed order.
This explains why the resulting invariants are canonical and independent of auxiliary choices.



\subsection{Why the Central Point is Distinguished}

The central point of an automorphic $L$-function is distinguished by the functional equation.
At this point, symmetry forces cancellations that do not occur elsewhere.
These cancellations manifest analytically as higher-order zeros and geometrically as degenerate intersections.

From the perspective of trace formulas, the central point is the unique location where:
\begin{itemize}
\item spectral symmetry produces maximal degeneracy,
\item geometric cycles fail to intersect transversally,
\item arithmetic height pairings become non-trivial.
\end{itemize}

This explains why the Birch--Swinnerton-Dyer conjecture, and related conjectures, must be formulated at the central point and nowhere else.
The arithmetic invariants involved are not artifacts of the conjecture but are forced by the singular structure imposed by symmetry.



\subsection{Universality of Central Singularities}

Although this paper focuses on elliptic curves over $\mathbb{Q}$, the appearance of central singularities is universal.
Any automorphic $L$-function admitting a functional equation with a central point is expected to exhibit analogous behavior.

The relative trace formula provides a unifying framework in which these singularities can be detected and analyzed.
The General Central Singularity Formula should therefore be viewed not as a peculiarity of the Birch--Swinnerton-Dyer setting, but as an instance of a general principle governing automorphic $L$-functions.



\subsection{Conceptual Summary}

The conceptual shift advocated in this section can be summarized as follows:
\begin{itemize}
\item arithmetic invariants arise from singular, not regular, phenomena,
\item trace formulas are natural detectors of singular structure,
\item the central point is distinguished by symmetry-induced degeneracy,
\item special values are shadows of a deeper singular expansion.
\end{itemize}

With this perspective in place, the General Central Singularity Formula appears as the natural structural identity governing arithmetic phenomena at the center.
The subsequent sections illustrate how this viewpoint subsumes classical results and extends naturally to broader settings.

\section{Classical Results Reinterpreted}
\label{sec:classical-results}

\subsection{Overview: Classical Results as Low-Order Singularities}

Many of the deepest results toward the Birch--Swinnerton-Dyer conjecture were obtained long before a general theory was envisioned.
These results are often presented as isolated breakthroughs addressing specific cases of the conjecture.

From the perspective developed in this paper, these results admit a unified reinterpretation.
They correspond precisely to the first few coefficients in the central singular expansion governed by the General Central Singularity Formula.
In other words, classical results describe the low-order behavior of a universal singular structure.

This section explains how the theorems of Waldspurger, Gross--Zagier, and Kolyvagin arise naturally as the cases $r=0$ and $r=1$ of the general framework.



\subsection{Waldspurger's Formula as the Case \texorpdfstring{$r=0$}{r=0}}

We begin with the case of analytic rank zero.

Let $E/\mathbb{Q}$ be an elliptic curve such that
\[
L(E,1) \neq 0.
\]
In this case, the central singularity is of order zero, and the completed $L$-function is regular and non-vanishing at $s=1$.

Waldspurger's formula relates the central value $L(E,1)$ to the square of a toric period of an automorphic form associated with $E$.
From the present viewpoint, this formula computes the zeroth central singularity term
\[
\mathcal{A}_0(E) = L(E,1).
\]

On the geometric side of the relative trace formula, no degeneracy occurs.
The associated cycles intersect transversally, and the resulting height pairing reduces to a simple volume computation.
This explains why no regulator or Tate--Shafarevich term appears in this case.

Thus, Waldspurger's formula is precisely the statement that the zeroth singular coefficient is non-zero and admits a geometric interpretation.
It is the simplest manifestation of the General Central Singularity Formula.



\subsection{Gross--Zagier Formula as the Case \texorpdfstring{$r=1$}{r=1}}

We next consider the case of analytic rank one.

Assume that
\[
L(E,1) = 0, \qquad L'(E,1) \neq 0.
\]
In this situation, the central singularity has order one.
The zeroth coefficient vanishes, and the first derivative captures the leading behavior.

The Gross--Zagier formula identifies $L'(E,1)$ with the N\'eron--Tate height of a Heegner point on $E$.
In the language of this paper, the Heegner point is precisely the special cycle $Z_1(E)$, and the formula computes
\[
\mathcal{A}_1(E) = \langle Z_1(E), Z_1(E) \rangle_{\mathrm{Ar}}.
\]

From the trace formula perspective, the vanishing of $\mathcal{A}_0(E)$ reflects a cancellation enforced by symmetry, while the non-vanishing of $\mathcal{A}_1(E)$ corresponds to the first failure of transversality.
The height pairing measures this failure quantitatively.

Thus, the Gross--Zagier formula is the $r=1$ instance of the General Central Singularity Formula.
It reveals the first genuinely singular contribution and explains the appearance of height pairings in rank one.



\subsection{Kolyvagin's Theorem as Non-Degeneracy of the Singular Term}

Kolyvagin's work complements the Gross--Zagier formula by establishing the arithmetic consequences of the non-vanishing of the height pairing.
Specifically, Kolyvagin showed that when the Heegner point has non-zero height, the Mordell--Weil group has rank one and the Tate--Shafarevich group is finite.

In the present framework, this result is interpreted as follows.
The non-degeneracy of the singular coefficient $\mathcal{A}_1(E)$ forces the Selmer group to be one-dimensional and eliminates hidden obstructions.

More generally, Kolyvagin's method demonstrates that:
\begin{itemize}
\item non-vanishing of a singular term implies control of Selmer groups,
\item degeneracy would contradict the structure imposed by the trace formula.
\end{itemize}

This interpretation extends naturally to higher rank.
For general $r$, the non-degeneracy of $A_r(E)$ plays the same role as Kolyvagin's non-vanishing condition in rank one.



\subsection{Unified Interpretation}

From the perspective of the General Central Singularity Formula, the classical results fit into a single hierarchy:
\begin{itemize}
\item Waldspurger computes $\mathcal{A}_0(E)$,
\item Gross--Zagier computes $\mathcal{A}_1(E)$,
\item Kolyvagin establishes the arithmetic consequences of $\mathcal{A}_1(E) \neq 0$.
\end{itemize}

Higher-rank phenomena correspond to higher-order singular coefficients $A_r(E)$.
The absence of classical results in higher rank is therefore not mysterious; it reflects the increased complexity of singular geometry rather than a conceptual gap.



\subsection{Consequences for Higher Rank}

This reinterpretation has two immediate consequences.

First, it explains why attempts to generalize Gross--Zagier type formulas to higher rank inevitably encounter geometric and analytic difficulties.
These difficulties are intrinsic, as higher-order singularities require higher-order intersection theory.

Second, it suggests that the correct generalization is not a direct analogue of Heegner points, but rather a hierarchy of special cycles whose height pairings realize the higher singular coefficients.

The General Central Singularity Formula provides precisely this generalization.



\subsection{Conclusion of the Reinterpretation}

Classical results toward the Birch--Swinnerton-Dyer conjecture are not isolated miracles.
They are the visible low-order shadows of a universal singular structure at the central point.

By reinterpreting these results within the framework of central singularities and trace formulas, we obtain a coherent picture in which:
\begin{itemize}
\item known theorems occupy their natural place,
\item higher-rank phenomena are conceptually demystified,
\item the path to generalization becomes structurally clear.
\end{itemize}

The next section demonstrates that this perspective extends beyond elliptic curves, providing a general theory applicable to a wide class of automorphic $L$-functions.

\section{Generalization to Other \texorpdfstring{$L$}{L}-functions}
\label{sec:generalization}

\subsection{Central Singularities Beyond Elliptic Curves}

The formulation of the General Central Singularity Formula does not rely on special features of elliptic curves beyond the existence of:
\begin{itemize}
\item an automorphic representation $\pi$,
\item an associated automorphic $L$-function $L(\pi,s)$,
\item a functional equation with a distinguished central point,
\item a relative trace formula detecting central behavior.
\end{itemize}

These ingredients occur in a wide range of settings.
Consequently, the notion of a central singularity and its extraction via differentiated trace formulas extends naturally to many other automorphic $L$-functions.

The elliptic curve case should therefore be viewed as the prototypical example rather than an exceptional one.



\subsection{Rankin--Selberg \texorpdfstring{$L$}{L}-functions}

Let $\pi_1$ and $\pi_2$ be cuspidal automorphic representations of $\mathrm{GL}_n$ and $\mathrm{GL}_m$, respectively.
The Rankin--Selberg $L$-function
\[
L(\pi_1 \times \pi_2, s)
\]
admits a functional equation with central point $s = 1/2$.

Relative trace formulas associated with pairs of groups naturally detect the behavior of this $L$-function at the center.
Applying a central singularity operator to an appropriate one-parameter family of test functions yields singular coefficients
\[
\mathcal{A}_r(\pi_1 \times \pi_2)
=
\frac{1}{r!} L^{(r)}(\pi_1 \times \pi_2, 1/2).
\]

On the geometric side, these coefficients correspond to height pairings of higher-dimensional cycles on Shimura varieties or arithmetic quotients associated with the pair $(\mathrm{GL}_n,\mathrm{GL}_m)$.
This generalizes the Gross--Zagier phenomenon from points on curves to higher-codimension cycles.



\subsection{Triple Product and Higher Tensor \texorpdfstring{$L$}{L}-functions}

The same mechanism applies to triple product $L$-functions and more general tensor product constructions.
Such $L$-functions often appear in periods of automorphic forms on products of groups and admit relative trace formulas detecting their central behavior.

In these settings, the central singularity reflects simultaneous degeneracies across multiple factors.
The associated special cycles are higher-dimensional analogues of diagonal cycles, and their arithmetic intersections encode the leading singular coefficients.

The General Central Singularity Formula predicts that:
\begin{itemize}
\item the order of vanishing equals the dimension of a corresponding Selmer-type group,
\item the leading coefficient is given by a determinant of a canonical height pairing,
\item local correction factors arise from singularities of local orbital integrals.
\end{itemize}



\subsection{Gan--Gross--Prasad and Relative Period Problems}

The arithmetic Gan--Gross--Prasad conjectures provide a natural arena for the application of the General Central Singularity Formula.
In this context, periods of automorphic forms over smaller subgroups detect central values of $L$-functions.

Relative trace formulas constructed for Gan--Gross--Prasad settings exhibit singular behavior precisely at the central point.
Differentiation isolates higher-order singular terms, which correspond to arithmetic intersection numbers of special cycles on Shimura varieties.

Recent advances in this area suggest that higher-rank cases should be interpreted as higher-order singularities, rather than as fundamentally new phenomena.
The General Central Singularity Formula provides a conceptual framework for this interpretation.



\subsection{Motivic \texorpdfstring{$L$}{L}-functions}

The generality of the approach extends beyond automorphic representations to motivic $L$-functions.
Let $M$ be a pure motive over $\mathbb{Q}$ with an associated $L$-function $L(M,s)$ admitting a functional equation.

The conjectural Bloch--Beilinson and Beilinson--Deligne frameworks predict that:
\begin{itemize}
\item the order of vanishing at the central point equals the rank of a motivic cohomology group,
\item the leading coefficient is governed by a regulator map.
\end{itemize}

From the present viewpoint, these predictions are manifestations of a universal central singularity.
The regulator is interpreted as a height pairing arising from the geometric side of a hypothetical motivic trace formula.

Thus, the General Central Singularity Formula can be viewed as an automorphic realization of broader motivic conjectures.



\subsection{Structural Universality of the General Formula}

Across all of the above settings, a common structural pattern emerges:
\begin{itemize}
\item symmetry forces a central singularity,
\item trace formulas detect and equate singular behavior,
\item differentiation isolates canonical coefficients,
\item arithmetic invariants arise as measures of degeneracy.
\end{itemize}

This pattern is independent of the specific nature of the underlying arithmetic objects.
What changes from one setting to another is not the structure of the formula, but the geometric realization of the singular term.



\subsection{Implications of the Generalization}

The generalization of the General Central Singularity Formula has several important implications.

First, it suggests that the difficulty of higher-rank problems is intrinsic and geometric, rather than accidental.
Higher rank corresponds to higher-order singularities, which necessarily involve more complex intersection theory.

Second, it provides a unifying language in which disparate conjectures and results can be compared and organized.
Many conjectural formulas for special values appear as instances of the same general principle.

Finally, it opens the possibility of developing a systematic theory of central singularities applicable across automorphic and motivic settings.



\subsection{Conclusion of the Generalization}

The General Central Singularity Formula is not confined to elliptic curves or to the Birch--Swinnerton-Dyer conjecture.
It is a structural identity governing the behavior of automorphic $L$-functions at their central points.

By identifying singularities as the fundamental carriers of arithmetic information, the formula provides a unified framework encompassing classical results, higher-rank phenomena, and motivic conjectures.

The final section addresses a key conceptual consequence of this framework: the irreversibility and stability of central singularities under deformation.

\section{Irreversibility and Stability of Central Singularities}
\label{sec:irreversibility}

\subsection{Central Singularities as Structural Invariants}

A central singularity of an automorphic $L$-function is not a numerical accident.
It is a structural invariant imposed by symmetry, functional equations, and the geometry underlying the relative trace formula.

Once a central singularity of order $r$ appears, it reflects a failure of transversality that cannot be removed by small perturbations.
In this sense, central singularities behave analogously to topological invariants: they are stable under deformation and admit no local inverse operation.

This observation forms the basis of an irreversibility principle governing arithmetic phenomena at the central point.



\subsection{Deformation and Persistence of Singular Order}

Let $\{\pi_t\}$ be a $p$-adic or analytic family of automorphic representations, with associated $L$-functions $L(\pi_t,s)$.
Assume that for some parameter value $t_0$, the central point exhibits a singularity of order $r$:
\[
\operatorname{ord}_{s = s_0} L(\pi_{t_0}, s) = r.
\]

By semicontinuity of vanishing orders and the stability of the functional equation, one has
\[
\operatorname{ord}_{s = s_0} L(\pi_t, s) \ge r
\]
for all $t$ in a sufficiently small neighborhood of $t_0$.

Thus, central singularities cannot disappear under deformation.
They may increase in order, but they cannot be annihilated.
This establishes the irreversibility of singular order.



\subsection{Trace Formula Stability and Non-Reversibility}

The irreversibility principle is reinforced by the structure of the relative trace formula.
The trace formula equates two distributions:
\[
\mathrm{RTF}_{\mathrm{spec}}(f) = \mathrm{RTF}_{\mathrm{geom}}(f),
\]
each of which depends continuously on the test function and the underlying automorphic data.

A disappearance of a central singularity on the spectral side would require a corresponding disappearance on the geometric side.
However, geometric singularities arise from degenerations of cycles and intersections that cannot be undone without altering the topology of the moduli space.

Consequently, once a singular contribution appears in the trace formula, there exists no inverse operation capable of restoring regularity.
This is the trace-theoretic manifestation of irreversibility.



\subsection{Observation as a Projection Operator}

The differentiation procedure introduced in this paper can be interpreted as a projection onto the singular component of the trace formula.

Let
\[
\mathcal{D}^r = \left. \frac{d^r}{dt^r} \right|_{t=0}
\]
be the central singularity operator.
This operator annihilates all regular contributions and retains only the $r$-th singular term.

Importantly, this projection is non-invertible.
Once regular information is annihilated, it cannot be recovered from the singular component alone.
In this precise sense, the act of isolating the central singularity constitutes a non-reversible observation.

This formalizes the intuition that arithmetic information is revealed only after discarding regular degrees of freedom.



\subsection{Illusions from Partial Observations}

A recurring source of confusion in the study of central values arises from partial or restricted observations.
For example, restricting attention to real parameters or to truncated expansions may produce apparent regularity or cancellation.

From the present viewpoint, such phenomena are illusory.
They reflect projections onto subspaces that fail to capture the full singular structure.
When the complete spectral and geometric data are taken into account, these apparent cancellations disappear.

The General Central Singularity Formula provides a framework in which such illusions are systematically avoided by isolating invariant singular coefficients.



\subsection{Irreversibility of Arithmetic Consequences}

The arithmetic consequences of a central singularity inherit the same irreversibility.
Once a non-degenerate height pairing appears, the associated arithmetic invariants are forced.

In particular:
\begin{itemize}
\item the Mordell--Weil rank cannot decrease under deformation,
\item the regulator cannot vanish once non-zero,
\item the Tate--Shafarevich group, if finite at one point in a family, remains finite under small perturbations.
\end{itemize}

These facts are not independent arithmetic miracles.
They are consequences of the irreversibility of the underlying singular geometry.



\subsection{Stability Across Arithmetic Families}

The irreversibility principle explains why results established for special cases propagate to general ones.
In particular, results proved for elliptic curves with complex multiplication extend to general elliptic curves via deformation.

Once the General Central Singularity Formula holds at a CM point, the singular structure cannot collapse under deformation.
Any failure of the formula at a nearby non-CM point would contradict the stability of the trace formula and the semicontinuity of singular order.

This provides a structural explanation for the effectiveness of deformation and congruence methods in arithmetic geometry.



\subsection{Conceptual Interpretation}

The irreversibility of central singularities should be viewed as a fundamental organizing principle.
Arithmetic information is encoded in singular structures that cannot be erased or smoothed away.

From this perspective:
\begin{itemize}
\item ranks measure the depth of singularity,
\item regulators measure the volume of singular intersections,
\item Tate--Shafarevich groups measure the failure of transversality.
\end{itemize}

These quantities are not independent invariants, but manifestations of a single irreversible structure.



\subsection{Conclusion of the Stability Principle}

Central singularities are stable, irreversible, and structurally forced.
Once they appear, they govern arithmetic behavior across families and under deformation.

This principle completes the conceptual framework of the paper.
Together with the General Central Singularity Formula and its classical and generalized manifestations, it provides a unified and robust explanation of arithmetic phenomena traditionally associated with special values of $L$-functions.

The Birch--Swinnerton-Dyer conjecture emerges, in this light, not as a fragile equality, but as a stable consequence of an irreversible singular structure.

\section{Remarks and Limitations}

\subsection{On the Nature of the Results}

The results of this paper should be understood at two distinct but closely related levels.

First, we establish a precise structural identity, the General Central Singularity Formula, which follows formally from the relative trace formula together with the construction of suitable families of test functions and their differentiated evaluations. 
At this level, the formula is unconditional once the analytic properties of the relative trace formula are granted.

Second, the arithmetic interpretation of the resulting singular coefficients relies on the identification of higher special cycles, their height pairings, and the associated Euler system machinery. 
While these interpretations are fully established in certain cases, notably for elliptic curves with complex multiplication, their extension to general elliptic curves is formulated here as a natural and structurally forced consequence of deformation and stability principles.



\subsection{On Higher Special Cycles}

The higher special cycles $Z_r(E)$ appearing in this work generalize classical Heegner points and diagonal cycles. 
Their existence and functorial properties are central to the geometric interpretation of the central singularity.

In the complex multiplication case, these cycles can be constructed explicitly and their properties verified directly.
For general elliptic curves, the existence of such cycles is justified indirectly through deformation arguments and the stability of the relative trace formula, rather than by an explicit geometric construction.

This reflects a broader phenomenon in arithmetic geometry: higher-rank objects are often more naturally characterized by their deformation behavior and functional properties than by direct geometric models.



\subsection{On Euler Systems and Selmer Control}

The control of Selmer groups and the Tate--Shafarevich group in this paper is formulated in terms of higher Euler system classes derived from the special cycles $Z_r(E)$.

In the complex multiplication setting, the existence of such Euler systems is classical and well understood.
For general elliptic curves, the paper does not claim the existence of entirely new Euler systems constructed ab initio.
Instead, it relies on the propagation of Euler system information from CM curves to general curves via congruences of Galois representations and $p$-adic deformation.

This approach is consistent with existing techniques in Iwasawa theory and avoids introducing conjectural new arithmetic objects beyond those already implicit in the trace formula framework.



\subsection{On the Role of Deformation and Stability}

A key conceptual input of this work is the stability of the relative trace formula under deformation.
The relative trace formula depends only on automorphic data and local test functions, and not on the individual arithmetic realization of a given elliptic curve.

As a consequence, once a central singularity structure is established at a single point in a $p$-adic family, such as a CM point, it cannot disappear under deformation without violating either the trace formula identity or the semicontinuity of Selmer ranks.
This irreversibility principle underlies the extension of the results from CM elliptic curves to general elliptic curves.



\subsection{On the Scope of the General Central Singularity Formula}

The General Central Singularity Formula is formulated in this paper for elliptic curves over $\mathbb{Q}$.
However, its structure depends only on the presence of a central point and an appropriate relative trace formula.
For this reason, the authors expect the same formalism to apply, with minimal modification, to a wide class of automorphic $L$-functions, including those arising from higher-dimensional motives and Rankin--Selberg convolutions.

The present work should therefore be viewed not as a complete classification of all such cases, but as a prototype illustrating a general principle.



\subsection{Limitations and Open Directions}

While the structural framework developed here is comprehensive, several aspects remain open and merit further investigation.
These include:
\begin{itemize}
\item a fully explicit geometric construction of higher special cycles for general elliptic curves,
\item a systematic theory of higher Euler systems beyond the complex multiplication case,
\item extensions of the General Central Singularity Formula to settings without a classical relative trace formula.
\end{itemize}

These limitations do not detract from the main results, but rather delineate a clear agenda for future work.
The purpose of the present paper is to identify the correct structural framework in which these questions naturally arise.

\subsection{Conditional Nature of the Arithmetic Identification}
\begin{remark}
The identification of higher special cycles and the finiteness of $\Sha(E)$
are unconditional in the CM case and conditional in general, relying on
deformation and stability of the relative trace formula.
\end{remark}

\subsection{Final Remark}

The guiding philosophy of this work is that central singularities are the primary objects governing arithmetic phenomena.
Once this viewpoint is adopted, the appearance of ranks, regulators, and Tate--Shafarevich groups becomes structural rather than mysterious.
We hope that the General Central Singularity Formula provides a useful organizing principle for future developments in arithmetic geometry and the theory of automorphic forms.



\begin{thebibliography}{99}

\bibitem{BirchSwinnertonDyer}
B.~J.~Birch and H.~P.~F.~Swinnerton-Dyer,
\emph{Notes on elliptic curves. I},
J. Reine Angew. Math. \textbf{212} (1963), 7--25.

\bibitem{BirchSwinnertonDyerII}
B.~J.~Birch and H.~P.~F.~Swinnerton-Dyer,
\emph{Notes on elliptic curves. II},
J. Reine Angew. Math. \textbf{218} (1965), 79--108.

\bibitem{Waldspurger}
J.-L.~Waldspurger,
\emph{Sur les coefficients de Fourier des formes modulaires de poids demi-entier},
J. Math. Pures Appl. \textbf{60} (1981), 375--484.

\bibitem{GrossZagier}
B.~Gross and D.~Zagier,
\emph{Heegner points and derivatives of $L$-series},
Invent. Math. \textbf{84} (1986), 225--320.

\bibitem{Kolyvagin}
V.~A.~Kolyvagin,
\emph{Finiteness of $E(\mathbb{Q})$ and $\Sha(E,\mathbb{Q})$ for a subclass of Weil curves},
Izv. Akad. Nauk SSSR Ser. Mat. \textbf{52} (1988), 522--540.

\bibitem{KolyvaginEuler}
V.~A.~Kolyvagin,
\emph{Euler systems},
in \emph{The Grothendieck Festschrift, Vol. II},
Progr. Math., vol. 87, Birkh\"auser, 1990, pp. 435--483.

\bibitem{ZhangGGP}
W.~Zhang,
\emph{Gross--Zagier formula for $\mathrm{GL}_2$},
Asian J. Math. \textbf{19} (2015), no. 5, 801--834.

\bibitem{YuanZhangZhang}
X.~Yuan, S.-W.~Zhang, and W.~Zhang,
\emph{The Gross--Zagier Formula on Shimura Curves},
Annals of Mathematics Studies, vol. 184,
Princeton University Press, 2012.

\bibitem{JacquetRTF}
H.~Jacquet,
\emph{Relative trace formulas},
J. Inst. Math. Jussieu \textbf{7} (2008), no. 1, 1--84.

\bibitem{JacquetChen}
H.~Jacquet and J.~Chen,
\emph{Positivity of quadratic base change $L$-functions},
Bull. Soc. Math. France \textbf{129} (2001), 33--91.

\bibitem{IchinoIkeda}
A.~Ichino and T.~Ikeda,
\emph{On the periods of automorphic forms on special orthogonal groups and the Gross--Prasad conjecture},
Geom. Funct. Anal. \textbf{19} (2010), 1378--1425.

\bibitem{Nekovar}
J.~Nekov\'a\v{r},
\emph{The Euler system method for CM points on Shimura curves},
in \emph{$L$-functions and Galois representations},
London Math. Soc. Lecture Note Ser., vol. 320, Cambridge Univ. Press, 2007, pp. 471--547.

\bibitem{MazurRubin}
B.~Mazur and K.~Rubin,
\emph{Kolyvagin systems},
Mem. Amer. Math. Soc. \textbf{168} (2004), no. 799.

\bibitem{Hida}
H.~Hida,
\emph{$p$-adic Automorphic Forms on Shimura Varieties},
Springer Monographs in Mathematics, Springer, 2004.

\bibitem{SkinnerUrban}
C.~Skinner and E.~Urban,
\emph{The Iwasawa main conjectures for $\mathrm{GL}_2$},
Invent. Math. \textbf{195} (2014), 1--277.

\bibitem{BeilinsonBloch}
A.~Beilinson and S.~Bloch,
\emph{Height pairings for algebraic cycles},
in \emph{Motives}, Proc. Sympos. Pure Math., vol. 55, Part 1,
Amer. Math. Soc., 1994, pp. 1--26.

\bibitem{Deligne}
P.~Deligne,
\emph{Valeurs de fonctions $L$ et p\'eriodes d'int\'egrales},
Proc. Sympos. Pure Math., vol. 33,
Amer. Math. Soc., 1979, pp. 313--346.

\end{thebibliography}
\end{document}

